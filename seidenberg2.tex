\section{Der zweite Cohen--Seidenbergsche Satz}

\subsection{Ganz abgeschlossene Integritätsbereiche}

\begin{frame}{Lokalisierung des ganzen Abschlusses}
	\begin{proposition}<+->
		Sei \(A \subset B\) eine Erweiterung kommutativer Ringe.
		Sei \(C\) der ganze Abschluß von \(A\) in \(B\). Sei \(S \subset A\)
		multiplikativ abgeschlossen. Dann ist \(S^{-1} C\) der ganze Abschluß
		von \(S^{-1} A\) in \(S^{-1} B\).
	\end{proposition}
	\begin{proof}<+->
		\begin{enumerate}[<+->]
		\item<.->
			Wir haben schon gesehen, daß \(S^{-1} C\) ganz über \(S^{-1} A\) ist.
		\item
			Sei umgekehrt \(\frac b s \in S^{-1} B\) ganz über \(S^{-1} A\),
			das heißt, wir haben eine Gleichung der Form
			\((\frac b s)^n + \frac {a_1}{s_1} (\frac b s)^{n - 1}
			+ \dotsb + \frac{a_n}{s_n} = 0\) mit \(a_i \in A, s_i \in S\).
		\item
			Sei \(t \coloneqq s_1 \dotsm s_n\). Multiplizieren wir die
			Ganzheitsbedingung mit \((st)^n\) erhalten wir eine
			Ganzheitsbedingung für \(bt\) über \(A\).
		\item
			Damit ist \(bt \in C\), also
			\(\frac b s = \frac{bt}{st} \in S^{-1} C\).
			\qedhere
		\end{enumerate}
	\end{proof}
\end{frame}

\begin{frame}{Ganz abgeschlossene Integritätsbereiche}
	\begin{definition}<+->
		Ein Integritätsbereich heißt \emph{ganz abgeschlossen}, falls er
		in seinem Quotientenkörper ganz abgeschlossen ist.
	\end{definition}
	\begin{example}<+->
		Der Ring \(\set Z\) der ganzen Zahlen ist ganz abgeschlossen.
	\end{example}
	\mode<article>{Ganz ähnlich wird gezeigt:}
	\begin{example}<+->
		Ein Polynomring \(K[x_1, \dotsc, x_n]\) in mehreren Variablen über einem
		Körper ist ganz abgeschlossen.
	\end{example}
\end{frame}

\begin{frame}{Lokalität der ganzen Abgeschlossenheit}
	\begin{proposition}<+->
		Sei \(A\) ein Integritätsbereich. Die folgenden Eigenschaften sind
		äquivalent:
		\begin{enumerate}[<+->]
		\item<.->
			Es ist \(A\) ganz abgeschlossen.
		\item
			Für jedes Primideal \(\ideal p\) in \(A\) ist \(A_{\ideal p}\)
			ganz abgeschlossen.
		\item
			Für jedes maximale Ideal \(\ideal m\) in \(A\) ist \(A_{\ideal m}\)
			ganz abgeschlossen.
		\end{enumerate}
	\end{proposition}
	\begin{proof}<+->
		Seien \(K\) der Quotientenkörper von \(A\) und \(C\) der ganze Abschluß
		von \(A\) in \(K\). Dann ist \(A\) genau dann ganz abgeschloseen,
		wenn \(\phi\colon A \to C, x \mapsto x\) surjektiv ist. Es ist
		Surjektivität eine lokale Eigenschaft und Bilden des ganzen Abschlusses
		mit Lokalisierung verträglich.
	\end{proof}
\end{frame}

\subsection{Ganzheit über Idealen}

\begin{frame}{Definition der Ganzheit über Idealen}
	\begin{definition}<+->
		Sei \(A \subset B\) eine Erweiterung kommutativer Ringe. Sei
		\(\ideal a\) ein Ideal von \(A\).
		\begin{enumerate}[<+->]
		\item<.->
			Ein Element \(x \in B\) heißt
			\emph{ganz über \(\ideal a\)}, falls \(x\) eine Gleichung der
			Form \(x^n + a_1 x^{n - 1} + \dotsb + a_n = 0\) mit \(a_i \in \ideal a\)
			erfüllt.
		\item
			Der \emph{ganze Abschluß von \(\ideal a\) in \(B\)} ist die Menge
			aller \(x \in B\), die ganz über \(\ideal a\) sind.
		\end{enumerate}
	\end{definition}
\end{frame}

\begin{frame}{Erweiterungen in ganzen Abschlüssen}
	\begin{lemma}<+->
		Sei \(A \subset B\) eine Erweiterung kommutativer Ringe. Sei \(C\) der
		ganze Abschluß von \(A\) in \(B\). Sei \(\ideal a\) ein Ideal von
		\(A\). Dann ist der ganze Abschluß von \(\ideal a\) in \(B\)
		das Wurzelideal \(\sqrt{C \ideal a}\).
	\end{lemma}
	\begin{proof}<+->
		\begin{enumerate}[<+->]
		\item<.->
			Sei \(x \in B\) mit \(x^n + a_1 x^{n - 1} + \dotsb + a_n = 0\)
			für gewisse \(a_i \in \ideal a\).
			Es folgt \(x \in C\) und damit \(x^n \in C \ideal a\), also
			\(x \in \sqrt{C \ideal a}\).
		\item
			Sei umgekehrt \(x \in \sqrt{C \ideal a}\), also \(x^n = \sum_i
			a_i x_i\) für \(a_i \in \ideal a\) und \(x_i \in C\). Es ist
			\(M \coloneqq A[x_1, \dotsc, x_n]\) ein endlich erzeugter
			\(A\)-Modul mit \(x^n M \subset \ideal a M\). Damit erfüllt
			die Multiplikation mit \(x^n\) auf \(M\) eine Ganzheitsbedingung
			über \(\ideal a\). Also ist \(x^n\) und damit auch \(x\) ganz über
			\(\ideal a\).
			\qedhere
		\end{enumerate}
	\end{proof}
\end{frame}

\begin{frame}{Eine Proposition über über einem Ideal ganze Elemente}
	\begin{proposition}<+->
		Seien \(A \subset B\) eine Erweiterung von Integritätsbereichen, und
		sei \(A\) ganz abgeschlossen mit Quotientenkörper \(K\). Sei weiter
		\(x \in B\) ganz über einem Ideal \(\ideal a\) von \(A\). Dann ist
		\(x\) algebraisch über \(K\) und für sein Minimalpolynom
		\(f \in K[t]\) gilt \(f \in \sqrt{\ideal a}[t]\).
	\end{proposition}
	\begin{proof}<+->
		Da \(x\) ganz über \(\ideal a\) ist, ist \(x\) insbesondere algebraisch
		über \(K\). Sei \(L\) ein Zerfällungskörper von \(f\),
		und seien \(x_1, \dotsc, x_n\) die Nullstellen (mit Vielfachheit)
		von \(f\). Jedes \(x_i\) erfüllt dieselbe Ganzheitsbedingung wie \(x\),
		daher sind die \(x_i\) ganz über \(\ideal a\).
		Die Koeffizienten von \(f\) sind Polynome in den \(x_i\), also
		ebenfalls ganz über \(\ideal a\). Da \(A\) ganz abgeschlossen ist,
		müssen sie daher in \(\sqrt{\ideal a}\) liegen.
	\end{proof}
\end{frame}

\subsection{Der zweite Cohen--Seidenbergsche Satz}

\begin{frame}{Der zweite Cohen--Seidenbergsche Satz}
	\begin{theorem}["`Going-down"']<+->
		\label{thm:going_down}
		Sei \(A \subset B\) eine ganze Erweiterung von Integritätsbereichen,
		und sei \(A\) ganz abgeschlossen. Dann gilt: 
		Seien \(\ideal p_1 \supset \dotsb
		\supset \ideal p_n\) eine Kette von Primidealen in \(A\) und
		\(\ideal q \colon \ideal q_1 \supset \dotsb \supset \ideal q_m\),
		\(m \le n\)
		eine Kette von Primidealen von \(B\) mit \(\ideal p_i = A \cap \ideal
		q_i\) für \(i \le m\).
		Dann kann die Kette \(\ideal q\) zu einer Kette \(\ideal q_1 \supset
		\dotsb \supset \ideal q_n\) mit \(\ideal p_i = A \cap \ideal q_i\)
		erweitert werden.
	\end{theorem}
\end{frame}

\begin{frame}{Beweis des zweiten Cohen--Seidenbergschen Satzes}
	\begin{proof}<+->
		\begin{enumerate}[<+->]
		\item<.->
			Es reicht, den Fall \(m = 1\), \(n = 2\) zu behandeln. Damit ist zu
			zeigen, daß \(\ideal p_2\) die Kontraktion eines Primideals in
			\(B_{\ideal q_1}\) ist, das heißt, daß
			\(A \cap B_{\ideal q_1} \ideal p_2 = \ideal p_2\).
		\item
			Jedes \(x \in B_{\ideal q_1} \ideal p_2\) ist von der Form
			\(x = \frac y s\) mit \(y \in B \ideal p_2\) und \(s \in
			B \setminus \ideal q_1\). Dann ist \(y\) ganz über \(\ideal p_2\),
			die Ganzheitsbedingung minimalen Grades von \(y\) über dem
			Quotientenkörper \(K\) von \(A\)
			ist also von der Form \(y^r + u_1 y^{r - 1} + \dotsb + u_r = 0\)
			mit \(u_i \in \ideal p_2\).
		\item
			Sei zusätzlich \(x \in A\). Dann ist \(s = y x^{-1}\) mit \(x^{-1}
			\in K\). Damit ist \(s^r + v_1 s^{r - 1} + \dotsb + v_r = 0\) mit
			\(v_i = u_i x^{-i}\) die Ganzheitsbedingung minimalen Grades für
			\(s\) über \(K\). Wir halten \(x^i v_i = u_i \in \ideal p_2\) fest.
		\item
			Da \(s\) ganz über \((1)\) in \(A\) ist, folgt \(v_i \in A\).
			Angenommen, \(x \notin \ideal p_2\). Da \(\ideal p_2\) prim ist,
			muß dann \(v_i \in \ideal p_2\) gelten.
		\item
			Daraus folgt wiederum, daß \(s^r \in B \ideal p_2 \subset
			B \ideal p_1 \subset \ideal q_1\), also \(s \in \ideal q_1\),
			ein Widerspruch.
			\qedhere
		\end{enumerate}
	\end{proof}
\end{frame}

\begin{frame}{Ganzer Abschluß in separablen Erweiterungen}
	\begin{proposition}<+->
		Sei \(A\) ein ganz abgeschlossener Integritätsbereich mit Quotientenkörper
		\(K\). Sei \(L\) eine endliche separable Körpererweiterung von \(K\) und
		\(B\) der ganze Abschluß von \(A\) in \(L\). Dann existiert eine
		Basis \(v_1, \dotsc, v_n\) von \(L\) über \(K\) so daß
		\(B \subset \sum\limits_{j = 1}^n A v_j\).
	\end{proposition}
\end{frame}

\begin{frame}{Beweis der Proposition}
	\begin{proof}<+->
		\begin{enumerate}[<+->]
		\item<.->
			Ist \(v \in L\), erfüllt \(v\) eine Gleichung
			\(a_0 v^r + a_1 v^{r - 1} + \dotsb + a_n = 0\) mit \(a_i \in A, a_0
			\neq 0\).
			Durch Multiplizieren dieser Gleichung mit \(a_0^{r - 1}\) sehen wir,
			daß \(a_0 v\) ganz über \(A\) und damit in \(B\) ist. Damit können
			wir eine beliebige Basis von \(L\) über \(K\) so umnormieren,
			daß wir eine Basis \((u_1, \dotsc, u_n)\) mit \(u_i \in B\) erhalten.
		\item
			Da \(L\) über \(K\) separabel ist, existieren \(v_1, \dotsc, v_n
			\in L\) mit \(\tr_{L/K}(u_i v_j) = \kron_{ij}\).
		\item
			Sei \(x \in B\), etwa \(x = \sum\limits_j x_j v_j\) mit \(x_j \in K\).
			Weiter ist \(x u_i \in B\), also \(\tr_{L/K} (x u_i) \in A\), da die
			Spur eines Elementes ein Vielfaches eines Koeffizienten seines
			Minimalpolynomes ist.
		\item
			Es gilt \(\tr_{L/K} (x u_i) = \sum\limits_j x_j \tr_{L/K}(u_i v_j)
			= x_i\), also \(x_i \in A\). Folglich ist \(B \subset \sum\limits_j
			A v_j\).
			\qedhere
		\end{enumerate}
	\end{proof}
\end{frame}

