\subsection{Artinsche Ringe}

\begin{exercise}
	Sei \(A\) ein noetherscher kommutativer Ring. Sei \((0) = \ideal q_1 \cap \dotsb \cap \ideal q_n\) eine
	minimale Primärzerlegung des Nullideals. Sei \(\ideal p_i \coloneqq \sqrt{\ideal q_i}\). 
	Mit \(\ideal p_i^{(r)}\) bezeichnen wir die \(r\)-te symbolische Potenz von \(\ideal p_i\) wie in
	\prettyref{exer:symbolic_power}. Zeige, daß für alle \(i\) ein \(r_i \in \set N_0\) mit
	\(\ideal p_i^{(r_i)} \subset \ideal q_i\) existiert.
	
	Im Falle, daß \(\ideal q_i\) eine isolierte Primärkomponente ist, ist \(A_{\ideal p_i}\) ein artinscher
	lokaler Ring mit maximalem Ideal \(\ideal m_i\). Wir haben daher \(\ideal m_i^r = 0\) für \(r \gg 0\).
	Daraus folgt \(\ideal q_i = \ideal p_i^{(r)}\) für \(r \gg 0\).
	
	Ist umgekehrt \(\ideal q_i\) eine eingebettete Primärkomponente, so ist \(A_{\ideal p_i}\) nicht artinsch,
	die Potenzen \(\ideal m_i^r\) sind also alle unterschiedlich, so daß ebenfalls die \(\ideal p_i^{(r)}\) unterschiedlich
	sind. Damit kann in der gegebenen Primärzerlegung \(\ideal q_i\) durch irgendeines der \(\ideal p_i\)-primären Ideale
	\(\ideal p_i^{(r)}\) mit \(r \ge r_i\) ersetzt werden, so daß es unendlich viele verschiedene
	minimale Primärzerlegungen des Nullideals gibt, welche sich nur in der \(\ideal p_i\)-Komponente
	unterscheiden.	
\end{exercise}

\begin{exercise}
	Seien \(F\) ein Körper und \(A\) eine endlich erzeugte kommutative \(F\)-Algebra. Zeige, daß die beiden
	folgenden Aussagen äquivalent sind:
	\begin{enumerate}
	\item
		Es ist \(A\) ein artinscher Ring.
	\item
		Es ist \(A\) eine endliche \(F\)-Algebra.
	\end{enumerate}
	
	(Tip: Um aus der ersten die zweite Aussage zu folgern, benutze \prettyref{thm:artin_structure}, um sich
	auf den Fall eines artinschen lokalen Ringes einschränken zu können. Nach \prettyref{cor:weak_hilbert1} ist
	der Restklassenkörper von \(A\) eine endliche Erweiterung von \(F\). Dann nutze aus, daß \(A\) als \(A\)-Modul
	endliche Länge hat.
	
	Um aus der zweiten die erste Aussage zu folgern, nutze aus, daß die Ideale in \(A\) unter anderem \(F\)-Vektorräume
	sind und daher die absteigende Kettenbedingung erfüllen.)
\end{exercise}

\begin{exercise}
	Sei \(A\) ein noetherscher kommutativer Ring. Seien \(\ideal p\) ein Primideal und \(\ideal q\) ein
	\(\ideal p\)-primäres Ideal in \(A\). Betrachte Ketten von Primäridealen von \(\ideal q\) nach \(\ideal p\).
	Zeige, daß die Länge dieser Ketten nach oben beschränkt ist und daß alle maximalen Ketten dieselbe Länge besitzen.
\end{exercise}

