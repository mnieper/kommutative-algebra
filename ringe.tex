\section{Ringe und Ringhomomorphismen}

\subsection{Ringe}

\begin{frame}{Definition eines Ringes}
    \begin{definition}
        Ein \emph{Ring \(A\)} ist eine Menge mit zwei ausgezeichneten Elementen,
        \(0\) und \(1\), und zwei binären Operationen \(+\) und \(\cdot\), so daß
        \pause
        \begin{enumerate}[<+->]
        \item
            \((A, +, 0)\) eine abelsche Gruppe ist,
        \item
            \((A, \cdot, 1)\) ein Monoid ist und
        \item
            die Multiplikation distributiv über die Addition ist, also
            \(x (y + z) = x y + x z\) und \((y + z) x = y x + z x\) für
            alle \(x, y, z \in A\).
        \end{enumerate}
        \visible<+->{%
            Der Ring heißt \emph{kommutativ}, falls \((A, \cdot, 1)\) ein
            kommutatives Monoid ist.}
    \end{definition}
    \begin{visibleenv}<+->
        Wir nennen \((A, +, 0)\) die \emph{abelsche Gruppe von \(A\)} und
        \((A, \cdot, 1)\) das \emph{multiplikative Monoid von \(A\)}.
    \end{visibleenv}
\end{frame}

\begin{frame}{Multiplikation mit Null}
    Sei \(A\) ein Ring.
    \begin{proposition}<+->
        Sei \(x \in A\). Dann ist \(0 \cdot x = 0\).
    \end{proposition}
    \begin{proof}<+->
        \(0 \cdot x = 0 \cdot x + 0 \cdot x - 0 \cdot x
        = (0 + 0) \cdot x - 0 \cdot x = 0 \cdot x - 0 \cdot x = 0.\)
    \end{proof}
    \begin{corollary}<+->
        Sei \(x \in A\). Dann ist \((-1) \cdot x = -x\).
    \end{corollary}
    \begin{proof}<+->
        \(x + (-1) \cdot x = 1 \cdot x + (-1) \cdot x
        = (1 - 1) \cdot x = 0 \cdot x = 0\).
    \end{proof}
\end{frame}

\begin{frame}{Explizite Beschreibung eines Ringes}
    Eine Menge \(A\) mit zwei ausgezeichneten Elementen \(0\) und \(1\) und
    zwei binären Operationen \(+\) und \(\cdot\) ist also genau dann ein Ring,
    falls für alle \(x, y, z \in A\) gilt:
    \begin{align*}
        x + (y + z) & = (x + y) + z ; \\
        y + x & = x + y; \\
        0 + x & = x; \\
        \exists (-x) \in A\colon -x + x & = 0; \\
        x (y z) & = (x y) z; \\
        1 \cdot x & = x; & x \cdot 1 & = x; \\
        x (y + z) & = x y + x z; & (y + z) x & = y x + z x; \\
        0 \cdot x & = 0; & x \cdot 0 & = 0.
    \end{align*}
\end{frame}

\begin{frame}{Beispiele von Ringen}   
    \begin{example}<+->
        Die ganzen Zahlen \(\set Z\) bilden einen kommutativen Ring. Dieser Ring
        ist grundlegend für die algebraische Zahlentheorie.
    \end{example}
    \begin{example}<+->
        Der Polynomring \(K[x_1, \dotsc, x_n]\) in \(n\) Variablen über einem
        Körper \(K\) ist ein kommutativer Ring. Diese Ringe sind grundlegend für
        die algebraische Geometrie.
    \end{example}
    \begin{visibleenv}<+->
        Bei der Definition eines Ringes ist \(0 = 1\) nicht ausgeschlossen worden:
        \begin{example}<+(-1)->
            Gilt \(0 = 1\) in einem Ring \(A\), folgt \(x = 1 \cdot x
            = 0 \cdot x = 0\) für alle \(x \in A\).
            Damit ist \(A = \{0\}\). Ein solcher Ring heißt der
            \emph{Nullring} und wird mit \(0\) bezeichnet.
        \end{example}
    \end{visibleenv}
\end{frame}

\subsection{Unterringe}

\begin{frame}{Definition eines Unterrings}
    \begin{definition}<+->
        Eine Teilmenge \(B\) eines Ringes \(A\) heißt \emph{Unterring von
        \(A\)}, falls \(B\) eine Untergruppe der abelschen Gruppe von \(A\)
        und ein Untermonoid des multiplikativen Monoides von \(A\) ist.
    \end{definition}
    Wir betrachten \(B\) auf offensichtliche Weise wieder als Ring.
    \\
    \begin{visibleenv}<+->
        Eine Teilmenge \(B\) von \(A\) ist also genau dann ein Unterring,
        falls für alle \(x, y \in B\) gilt:
        \begin{align*}
            x + y & \in B; \\
            -x & \in B; \\
            0 & \in B; \\
            x y & \in B; \\
            1 & \in B.
        \end{align*}
    \end{visibleenv}
\end{frame}

\begin{frame}{Beispiele für Unterringe}
    \begin{example}<+->
        Die ganzen Zahlen \(\set Z\) bilden einen Unterring der rationalen Zahlen
        \(\set Q\).
    \end{example}
    \begin{example}<+->
        Wir können einen Körper \(K\) als Unterring des Ringes \(K[x]\) der Polynome
        in einer Variablen über \(K\) auffassen.
    \end{example}
    \begin{example}<+->
        Der (bezüglich der Inklusionsordnung) größte Unterring eines Ringes \(A\)
        ist der Ring \(A\) selbst.
    \end{example}
\end{frame}

\begin{frame}{Der von einer Teilmenge erzeugte Unterring}
    \begin{proposition}<+->
        Ist \((B_i)_{i \in I}\) eine Familie von Unterringen eines
        Ringes \(A\),
        so ist der Schnitt \(\bigcap\limits_{i \in I} B_i\) ein Unterring von \(A\).
    \end{proposition}
    \begin{proof}<+->
        Da die \(B_i\) jeweils \(0\) und \(1\) enthalten und jeweils abgeschlossen
        unter Addition und Multiplikation sind, folgt, daß auch der Schnitt \(0\)
        und \(1\) enthält und abgeschlossen unter Addition und Multiplikation ist.
    \end{proof}
    \begin{construction}<+->
        Sei \(S\) eine Teilmenge eines Ringes \(A\). Dann ist der Schnitt
        aller Unterringe von \(A\), welche \(S\) enthalten, der kleinste Unterring
        von \(A\), welcher \(S\) enthält.
        \qed
    \end{construction}        
\end{frame}

\subsection{Ringhomomorphismen}

\begin{frame}{Definition eines Ringhomomorphismus}
    \begin{definition}<+->
        Ein \emph{Ringhomomorphismus \(\phi\)} ist eine Abbildung
        \(\phi\colon A \to B\) zwischen zwei Ringen, welche einen Homomorphismus
        zwischen den abelschen Gruppen und einen Homomorphismus zwischen den
        multiplikativen Monoiden von \(A\) und \(B\) induziert.
    \end{definition}
    \begin{visibleenv}<+->
        Eine Abbildung \(\phi\colon A \to B\) ist also genau dann ein
        Ringhomomorphismus, falls für alle \(x, y \in A\) gilt:
        \begin{align*}
        \phi(x + y) & = \phi(x) + \phi(y); \tag{$*$}
        \\
        \phi(0)     & = 0 && \text{(folgt schon aus ($*$))}; \\
        \phi(x y)   & = \phi(x) \phi(y); \\
        \phi(1)     & = 1.
        \end{align*}
    \end{visibleenv}
\end{frame}

\begin{frame}{Beispiele von Ringhomomorphismen}
    \begin{example}<+->
        Die Identität \(\id_A\) eines Ringes ist ein Ringhomomorphismus.
    \end{example}
    \begin{example}<+->
        Die Inklusion \(B \injto A\) eines Unterrings ist ein Ringhomomorphismus.
    \end{example}
    \begin{proposition}<+->
        Seien \(\phi\colon A \to B\) und \(\psi\colon B \to C\)
        Ringhomomorphismen. Dann ist \(\psi \circ \phi\colon A \to C\) ein
        Ringhomomorphismus.
    \end{proposition}
    \begin{proof}<+->
        Die Verknüpfung zweier Gruppen- bzw. Monoidhomomorphismen ist wieder
        ein Gruppen- bzw.~Monoidhomomorphismus.
    \end{proof}
\end{frame}

\begin{frame}{Abbildungen als Ringhomomorphismen}
    \begin{visibleenv}<+->
        Seien \(A\) und \(B\) zwei Ringe und \(\phi\colon A \to B\) eine Abbildung.
    \end{visibleenv}
    \begin{proposition}<+->
        Sei \(\pi\colon C \surjto A\) ein surjektiver Ringhomomorphismus. Ist dann
        \(\phi \circ \pi\colon C \to B\) ein Ringhomomorphismus, so auch \(\phi\colon A \to B\).
    \end{proposition}
    \begin{proof}<+->
        Seien \(a, a' \in A\). Dann existieren \(c, c' \in C\) mit \(\pi(c) = a\)
        und \(\pi(c') = a'\).
        Es folgt \(\phi(a a') = \phi(\pi(c) \pi(c')) = \phi(\pi(c c')) = \phi(\pi(c))
        \phi(\pi(c')) = \phi(a) \phi(a')\). Analog erhält \(\phi\) auch Addition,
        \(0\) und \(1\).
    \end{proof}
    \begin{proposition}<+->
        Sei \(\iota\colon B \injto C\) ein injektiver Ringhomomorphismus. Ist dann
        \(\iota \circ \phi\colon A \to C\) ein Ringhomomorphismus, so auch \(\phi\colon A \to B\).
        \qed
    \end{proposition}
\end{frame}

\begin{frame}{Ringisomorphismen}
    \begin{visibleenv}<+->
        Sei \(\phi\colon A \to B\) ein Ringhomomorphismus.
    \end{visibleenv}    
    \begin{definition}<.->
        Der Homomorphismus \(\phi\) heißt \emph{Isomorphismus}, falls
        ein Ringhomomorphismus \(\check\phi\colon B \to A\) existiert, so daß
        \(\check \phi \circ \phi = \id_A\) und \(\phi \circ \check \phi = \id_B\).
    \end{definition}
    \begin{proposition}<+->
        Ist \(\phi\colon A \to B\) eine bijektive Abbildung, so ist \(\phi\) schon ein 
        Isomorphismus.
    \end{proposition}
    \begin{proof}<+->
        Da \(\id_B = \phi \circ \phi^{-1}\) ein Ringhomomorphismus ist und
        \(\phi\) ein injektiver Ringhomomorphismus ist, ist auch \(\phi^{-1}\)
        ein Ringhomomorphismus.
    \end{proof}
\end{frame}

\subsection{Bezeichnungen}

\begin{frame}{Bezeichnungen von Ringen, Ringelementen und Ringhomomorphismen}
    \begin{convention}<+->
        Ringe bezeichnen wir mit großen lateinischen Buchstaben
        \(A, B, C, \dotsc\), Ringelemente mit kleinen lateinischen Buchstaben
        \(a, b, c, f, g, h, x, y, z, \dotsc\).
    \end{convention}
    \begin{visibleenv}<+->
        Betrachten wir einen Ring als Rechenbereich, nennen wir seine Elemente
        auch \emph{Zahlen}. Betrachten wir einen Ring als Algebra von Funktionen auf
        einem Raum, heißen die Ringelemente entsprechend \emph{Funktionen}.
    \end{visibleenv}
    \begin{convention}<+->
        Ringhomomorphismen bezeichnen wir mit kleinen griechischen
        Buchstaben \(\phi, \psi, \dotsc\).
    \end{convention}
\end{frame}

