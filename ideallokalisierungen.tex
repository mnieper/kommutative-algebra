\section{Idealerweiterungen und -kontraktionen in Lokalisierungen}

\subsection{Erweiterungen und Kontraktionen}

\begin{frame}{Erweiterung in einer Lokalisierung}
	Sei \(A\) ein kommutativer Ring. Sei \(S \subset A\) multiplikativ abgeschlossen.
	\begin{example}<+->
		Sei \(\ideal a\) ein Ideal in \(A\).
		Dann gilt für die Erweiterung \((S^{-1} A) \ideal a = S^{-1} \ideal a\), denn jedes Element in \((S^{-1} A) \ideal a\) ist
		von der Form \(\sum_i \frac {a_i}{s_i}\) mit \(a_i \in \ideal a\) und \(s_i \in S\), und Ausdrücke dieser Form
		können wir auf einen gemeinsamen Nenner bringen.
	\end{example}
	\begin{proposition}<+->
		Alle Ideale in \(S^{-1} A\) sind erweiterte Ideale, also von der Form \(S^{-1} \ideal a\).
	\end{proposition}
	\begin{proof}<+->
		Sei \(\ideal b\) ein Ideal in \(S^{-1} A\). Sei \(\frac x s \in \ideal b\). Dann ist \(\frac x 1\) in \(\ideal b\),
		also \(x \in A \cap \ideal b\). Damit folgt \(\frac x s \in S^{-1}(A \cap \ideal b)\).
	\end{proof}
\end{frame}

\begin{frame}{Kontraktionen von Erweiterungen in einer Lokalisierung}
	Sei \(A\) ein kommutativer Ring. Seien \(S \subset A\) multiplikativ abgeschlossen und \(\ideal a\) ein Ideal in \(A\).
	\begin{proposition}<+->
		Es ist \(A \cap (S^{-1} \ideal a) = \bigcup\limits_{u \in S} (\ideal a : u)\). 
	\end{proposition}
	\begin{proof}<+->
		\(x \in A \cap (S^{-1} \ideal a) \iff \exists a \in \ideal a \exists s \in S \colon \frac x 1 = \frac a s
		\iff \exists a \in \ideal a \exists s, t \in S \colon (x s - a) t = 0 \iff \exists u \in S\colon
		ux \in \ideal a \iff x \in \bigcup\limits_{u \in S} (\ideal a : u)\).
	\end{proof}	
	\begin{example}<+->
		Es \(S^{-1} \ideal a = (1)\) genau dann, wenn \(\ideal a \cap S \neq \emptyset\).
	\end{example}
\end{frame}
	
\begin{frame}{Kontrahierte Ideale und Lokalisierungen}
	\begin{proposition}<+->
		Sei \(A\) ein kommutativer Ring. Seien \(S \subset A\) multiplikativ abgeschlossen und \(\ideal a\) ein Ideal in \(A\).
		Es ist \(\ideal a\) genau dann ein kontrahiertes Ideal bezüglich \(A \to S^{-1} A\), also ein Ideal der Form
		\(A \cap \ideal b\), \(\ideal b \subset S^{-1} A\),
		wenn kein Element von \(S\) ein Nullteiler in \(A/\ideal a\) ist.
	\end{proposition}
	\begin{proof}<+->
		\(\ideal a\) ist genau dann kontrahiert, wenn \(A \cap (S^{-1} \ideal a) \subset \ideal a\), wenn also aus \(s x \in
		\ideal a\) mit \(s \in S, x \in A\) schon \(x \in \ideal a\) folgt. Das ist wiederum gleichbedeutend damit, daß
		kein \(s \in S\) ein Nullteiler in \(A/\ideal a\) ist.
	\end{proof}
\end{frame}

\begin{frame}{Lokalisierungen und endliche Summen, Produkte und Schnitte und Wurzeln von Idealen}
	\begin{proposition}<+->
		Sei \(A\) ein kommutativer Ring. Die Lokalisierung nach einer multiplikativ abgeschlossenen
		Teilmenge \(S \subset A\) kommutiert mit folgenden Operationen auf Idealen: endliche Summen,
		endliche Produkte, endliche Schnitte und Wurzeln.
	\end{proposition}
	\begin{proof}<+->
		\begin{enumerate}[<+->]
		\item<.->
			Endliche Summen und Produkte vertauschen mit Idealerweiterungen, und die Lokalisierung
			eines Ideals nach \(S\) entspricht der Idealerweiterung in \(S^{-1} A\).
		\item
			Endliche Schnitte von Idealen vertauschen mit Lokalisierung, denn dies gilt allgemeiner für Untermoduln.
		\item
			Sei \(\ideal a \subset A\) ein Ideal. Daß \(S^{-1} \sqrt{\ideal a} \subset \sqrt{S^{-1} \ideal a}\)
			folgt aus allgemeinen Eigenschaften der Erweiterung von Idealen. Die andere Inklusion ist einfach.
			\qedhere
		\end{enumerate}		
	\end{proof}
\end{frame}

\begin{frame}{Nilradikal der Lokalisierung}
	Sei \(A\) ein kommutativer Ring. Sei \(S \subset A\) multiplikativ abgeschlossen.
	\begin{corollary}<+->
		Ist \(\ideal n\) das Nilradikal von \(A\), so ist \(S^{-1} \ideal n\) das Nilradikal von \(S^{-1} A\).
		\qed
	\end{corollary}
\end{frame}

\subsection{Primideale und Lokalisierungen}

\begin{frame}{Primideale in Lokalisierungen}
	\begin{proposition}<+->
		Sei \(A\) ein kommutativer Ring. Sei \(S \subset A\) multiplikativ abgeschlossen.
		Durch \(\ideal q = S^{-1} \ideal p\) ist eine bijektive,
		ordnungserhaltende Korrespondenz zwischen den Primidealen
		\(\ideal p\) von \(A\) mit \(\ideal p \cap S = \emptyset\) und den Primidealen \(\ideal q\) von \(S^{-1} A\) gegeben.
	\end{proposition}
	\begin{proof}<+->
		\begin{enumerate}[<+->]
		\item<.->
			Ist \(\ideal q\) ein Primideal in \(S^{-1} A\), so ist \(\ideal p = A \cap \ideal q\) als Kontraktion eines
			Primideals wieder ein Primideal.
		\item
			Ist \(\ideal p\) ein Primideal in \(A\), so ist \(A/\ideal p\) ein Integritätsbereich. Ist \(\bar S\) das Bild von
			\(S\) in \(A/\ideal p\), so ist \(S^{-1} A/S^{-1} \ideal p \cong \bar S^{-1}(A/\ideal p)\) und damit entweder der Nullring
			oder wieder ein Integritätsbereich.
		\item
			Der erste Fall tritt genau dann ein, wenn \(S^{-1} \ideal p = (1)\), also genau dann, wenn
			\(\ideal p \cap S \neq \emptyset\).
			Der zweite Fall tritt genau dann ein, wenn \(S^{-1} \ideal p\) ein Primideal ist.
			\qedhere
		\end{enumerate}
	\end{proof}
\end{frame}

\begin{frame}{Bemerkungen zu Lokalisierungen und Idealen}
	\begin{remark}<+->
		Sei \(A\) ein kommutativer Ring. Daß zu einem nicht nilpotenten Element \(f \in A\) ein Primideal \(\ideal p\) von
		\(A\) mit \(f \notin \ideal p\) existiert, kann mittels Lokalisierung so bewiesen werden: Da \(f\) nicht nilpotent
		ist, ist \(A[f^{-1}] \neq 0\). Damit besitzt \(A[f^{-1}]\) ein maximales Ideal \(\ideal m\). Es folgt, daß
		\(\ideal p = A \cap \ideal m\) ein Primideal mit \(f \notin \ideal p\) ist.
	\end{remark}
\end{frame}

\begin{frame}{Primideale in Halmen}
	Sei \(\ideal p\) ein Primideal in einem kommutativen Ring \(A\).
	\begin{corollary}<+->
		Durch \(\ideal r = A_{\ideal p} \ideal q\) ist eine bijektive, ordnungserhaltende Korrespondenz zwischen den
		Primidealen \(\ideal q\) von \(A\) mit \(\ideal q \subset \ideal p\) und den Primidealen \(\ideal r\) im Halm \(A_{\ideal p}\)
		gegeben.
		\qed
	\end{corollary}
	\begin{remark}<+->
		Sei \(\ideal q \subset \ideal p\) ein weiteres Primideal.
		Lokalisieren bei \(\ideal p\) schneidet also alle Primideale außer denen heraus, die in \(\ideal p\) enthalten sind.
		\\
		Auf der anderen Seite schneidet der Wechsel von \(A\) nach \(A/\ideal q\) alle Primideale außer denen heraus, die
		\(\ideal q\) enthalten.
		\\
		Beim Übergang von \(A\) auf den Ring \(A_{\ideal p}/(A_{\ideal p} \ideal q) = (A/\ideal q)_{\bar{\ideal p}}\)
		beschränken wir unsere Betrachtung also auf alle Primideale zwischen \(\ideal q\) und \(\ideal p\) beschränken.
		Im Spezialfall \(\ideal p = \ideal q\) erhalten wir den \emph{Restklassenkörper \(A(\ideal p)\) von \(A\) an \(\ideal p\)},
		also den Restklassenkörper des Halmes \(A_{\ideal p}\) beziehungsweise den Quotientenkörper des Quotienten \(A/\ideal p\).
	\end{remark}
\end{frame}

\begin{frame}{Kontraktionen von Primidealen in beliebigen Ringhomomorphismen}
	\begin{proposition}<+->
		Sei \(\phi\colon A \to B\) ein Homomorphismus kommutativer Ringe. Dann ist ein Primideal \(\ideal p\) in
		\(A\) genau dann eine Kontraktion eines Primideals von \(B\), falls \(\ideal p = A \cap (B\ideal p)\).
	\end{proposition}
	\begin{proof}<+->
		\begin{enumerate}[<+->]
		\item<.->
			Ist \(\ideal p = A \cap \ideal q\) für ein Primideal \(\ideal q\) von \(B\), so ist \(A \cap (B \ideal p)
			= A \cap (B (A \cap \ideal q)) = A \cap \ideal q = \ideal p\).
		\item
			Sei umgekehrt \(\ideal p = A \cap (B \ideal p)\).
			Sei \(S\) das Bild von \(A \setminus \ideal p\) in \(B\).
			Dann ist \(S \cap (B \ideal p) = \emptyset\), also ist \(S^{-1} (B \ideal p)\) in einem maximalen
			Ideal \(\ideal m\) in \(S^{-1} B\) enthalten.
		\item
			Sei \(\ideal q \coloneqq B \cap \ideal m\). Es folgt, daß \(\ideal q\) ein Primideal ist. Weiter ist
			\(\ideal q \supset B \ideal p\) und \(\ideal q \cap S = \emptyset\). Damit ist \(A \cap \ideal q = \ideal p\).
			\qedhere
		\end{enumerate}
	\end{proof}
\end{frame}

\subsection{Lokalisierungen und der Annulator}

\begin{frame}{Lokalisierung des Annulators}
	\begin{proposition}<+->
		Sei \(A\) ein kommutativer Ring. Sei \(M\) ein endlich erzeugter \(A\)-Modul. Sei \(S \subset A\) multiplikativ abgeschlossen.
		Dann ist \(S^{-1} \ann(M) = \ann(S^{-1} M)\).
	\end{proposition}
	\begin{proof}<+->
		\begin{enumerate}[<+->]
		\item<.->
			Sei die Aussage wahr für zwei Untermoduln \(N\) und \(P\) von \(M\). Dann gilt sie auch für \(N + P\):
			\(S^{-1} \ann(N + P) = S^{-1} (\ann N \cap \ann P) = S^{-1} \ann N \cap S^{-1} \ann P
			= \ann(S^{-1} N) \cap \ann(S^{-1} P) = \ann(S^{-1} N + S^{-1} P) = \ann(S^{-1} (N + P))\).
		\item
			Damit reicht es die Aussage für einen Modul der Form \(M = A/\ideal a\), wobei \(\ideal a\) ein Ideal in \(A\) ist,
			zu beweisen. Es ist \(\ideal a = \ann(M)\). Weiter ist \(S^{-1} M \cong (S^{-1} A)/(S^{-1} \ideal a)\), also
			\(\ann(S^{-1} M) = S^{-1} \ideal a\).
			\qedhere
		\end{enumerate}
	\end{proof}
\end{frame}

\begin{frame}{Lokalisierungen und Quotienten}
	\begin{corollary}<+->
		Sei \(A\) ein kommutativer Ring. Seien \(N, P\) zwei Untermoduln eines \(A\)-Moduls \(M\). Sei \(S \subset A\) multiplikativ abgeschlossen.
		Ist \(P\) endlich erzeugt, so gilt \(S^{-1} (N : P) = (S^{-1} N : S^{-1} P)\).
	\end{corollary}
	\begin{proof}<+->
		Es ist \((N : P) = \ann((N + P)/N)\).
	\end{proof}
	\begin{example}<+->
		Seien \(\ideal a, \ideal b\) Ideale von \(A\). Ist \(\ideal b\) endlich erzeugt, so gilt
		\(S^{-1} (\ideal a : \ideal b) = (S^{-1} \ideal a : S^{-1} \ideal b)\).
	\end{example}
\end{frame}

