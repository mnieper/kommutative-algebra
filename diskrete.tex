\section{Diskrete Bewertungsringe}

\subsection{Eindimensionale noethersche Integritätsbereiche}

\begin{frame}{Eindeutige Faktorisierung in Primärideale}
	\begin{proposition}<+->
		Sei \(A\) ein eindimensionaler noetherscher Integritätsbereich. Dann kann
		jedes Ideal \(\ideal a \neq (0)\) von \(A\) eindeutig als Produkt primärer
		Ideale mit paarweise verschiedenen Wurzelidealen geschrieben werden.
	\end{proposition}
	\begin{proof}<+->
		\begin{enumerate}[<+->]
		\item<.->
			Da \(A\) noethersch ist, existiert eine  minimale Primärzerlegung
			\(\ideal a = \bigcap\limits_{i = 1}^n \ideal q_i\). Die \(\ideal p_i
			\coloneqq \sqrt{\ideal q_i}\) sind Primideale ungleich dem Nullideal, welches
			selbst ein Primideal ist. Wegen \(\dim A = 1\) sind die
			\(\ideal p_i\) paarweise verschiedene maximale Ideale.
		\item
			Es folgt, daß die \(\ideal q_i\) paarweise koprim sind, also
			\(\ideal a = \bigcap\limits_i \ideal q_i = \prod\limits_i \ideal q_i\).
		\item
			Ist umgekehrt \(\ideal a = \prod\limits_i \ideal q_i\) mit
			\(\ideal p_i\)-primären Idealen \(\ideal q_i\), so folgt analog \(\ideal a = \bigcap\limits_i \ideal q_i\).
			Dies ist eine minimale Primärzerlegung von \(\ideal a\), in dem die
			\(\ideal q_i\) isolierte Komponenten sind.
			\qedhere
		\end{enumerate}
	\end{proof}
\end{frame}

\begin{frame}{Eindeutige Faktorisierung in Primideale}
	\begin{remark}<+->
		Ist \(A\) ein eindimensionaler noetherscher Integritätsbereich, in dem jedes Primärideal
		Potenz eines Primideals ist, so folgt aus der Proposition, daß jedes nicht verschwindende
		Ideal in \(A\) eine eindeutige Faktorisierung in Primideale besitzt.
		\\
		Für jedes Primideal \(\ideal p \neq (0)\) besitzt der Halm \(A_{\ideal p}\) dann dieselben
		Eigenschaften, in \(A_{\ideal p}\) kann jedes nicht verschwindende Ideal also als eindeutige
		Potenz des maximalen Ideals geschrieben werden.
	\end{remark}
\end{frame}

\subsection{Diskrete Bewertungsringe}

\begin{frame}{Definition einer diskreten Bewertung}
	\begin{definition}<+->
		Eine \emph{diskrete Bewertung auf einem Körper \(K\)} ist eine surjektive Abbildung
		\(\nu\colon K \surjto \set Z \cup \{\infty\}\) mit folgenden Eigenschaften:
		\begin{enumerate}[<+->]
		\item<.->
			\(\nu(x) = \infty \iff x = 0\) für alle \(x \in K\).
		\item
			\(\nu(xy) = \nu(x) + \nu(y)\) für alle \(x, y \in K\).
		\item
			\(\nu(x + y) \ge \min(\nu(x), \nu(y))\) für alle \(x, y \in K\).
		\end{enumerate}
	\end{definition}
	\begin{visibleenv}<+->
		Die Menge aller \(x \in K\) mit \(\nu(x) \ge 0\) ist ein Unterring von \(K\)
		und heißt der \emph{Bewertungsring von \(K\) (zu \(\nu\))}. Dieser ist in der Tat ein
		Bewertungsring.
	\end{visibleenv}
\end{frame}

\begin{frame}{Beispiele diskreter Bewertungen}
	\begin{example}<+->
		Sei \(p\) eine Primzahl. Jedes \(x \in \set Q^\units\) kann eindeutig in der Form
		\(x = p^a y\) geschrieben werden, wobei \(a \in \set Z\) ist und \(y\) ein Bruch ist,
		dessen Zähler und Nenner nicht durch \(p\) teilbar sind.
		\\
		Durch die Setzung \(\nu(x) \coloneqq a\) wird eine diskrete Bewertung auf \(\set Q\) definiert,
		deren Bewertungsring der lokale Ring \(\set Z_{(p)}\) ist.
	\end{example}
	\begin{example}<+->
		Seien \(F\) ein Körper und \(K \coloneqq F(x)\) der Körper der rationalen Funktion über \(F\) in \(x\).
		Sei \(f \in F[x]\) ein irreduzibles Polynom. Jedes \(g \in K^\units\) kann eindeutig in der
		Form \(g = f^a h\) geschrieben werden, wobei \(a \in \set Z\) ist und \(h\) ein Bruch ist,
		dessen Zähler und Nenner nicht durch \(f\) teilbar sind.
		\\
		Durch die Setzung \(\nu(g) \coloneqq a\) wird eine diskrete Bewertung auf \(K\) definiert,
		deren Bewertungsring der lokale Ring \(F[x]_{(f)}\) ist.
	\end{example}
\end{frame}

\begin{frame}{Definition eines diskreten Bewertungsringes}
	\begin{definition}<+->
		Ein Integritätsbereich \(A\) heißt \emph{diskreter Bewertungsring}, falls
		\(A\) der Bewertungsring einer diskreten Bewertung auf seinem Quotientenkörper \(K\) ist.
	\end{definition}
	\begin{visibleenv}<+->
		Ist \(A\) ein diskreter Bewertungsring zur Bewertung \(\nu\colon K \to \set Z \cup \{\infty\}\),
		so ist \(A\) insbesondere ein lokaler Ring mit maximalem Ideal \(\ideal m \coloneqq
		\{x \in A \mid \nu(x) > 0\}\).
	\end{visibleenv}
\end{frame}

\begin{frame}{Hauptideale in einem diskreten Bewertungring}
	\begin{proposition}<+->
		Sei \(A\) ein diskreter Bewertungsring mit Bewertung \(\nu\colon K \to \set Z \cup \{\infty\}\).
		Seien \(x, y \in A\). Dann gilt \(\nu(x) = \nu(y)\) genau dann, wenn \((x) = (y)\).
	\end{proposition}
	\begin{proof}<+->
		Aus \(\nu(xy^{-1}) = 0\) folgt, daß \(xy^{-1}\) eine Einheit in \(A\) ist.
		\\
		Umgekehrt genauso.
	\end{proof}
\end{frame}

\begin{frame}{Diskrete Bewertungsringe sind noethersch}
	Sei \(A\) ein diskreter Bewertungsring mit Bewertung \(\nu\colon K \to \set Z \cup \{\infty\}\).
	\begin{proposition}<+->
		Jedes Ideal in \(A\) ist von der Form \(\ideal m_k = \{x \in A \mid \nu(x) \ge k\}\) mit
		\(k \in \set N_0 \cup \{\infty\}\).
	\end{proposition}
	\begin{proof}<+->
		Zu einem Ideal \(\ideal a\) sei \(k \coloneqq \inf\{\nu(x) \mid x \in \ideal a\}\). Dann gilt
		\(\ideal a = \ideal m_k\), denn ist \(x \in \ideal a\), so ist auch \(y \in \ideal a\), falls
		\(\nu(y) \ge \nu(x)\).
	\end{proof}
	\begin{corollary}<+->
		Diskrete Bewertungsringe sind noethersche lokale Ringe.
	\end{corollary}
	\begin{proof}<+->
		Die einzigen Ideale in \(A\) bilden die absteigende Folge \((1) = \ideal m_0 \supset \ideal m_1
		\supset \ideal m_2 \supset \dotsb\).
	\end{proof}
\end{frame}

\begin{frame}{Diskrete Bewertungsringe sind eindimensional}
	\begin{proposition}<+->
		Sei \(A\) ein diskreter Bewertungsring.
		Dann ist \(A\) ein eindimensionaler noetherscher lokaler Integritätsbereich,
		in dem jedes (nicht verschwindende) Ideal eine Potenz des
		maximalen Ideals \(\ideal m\) ist.
	\end{proposition}
	\begin{proof}<+->
		Da die Bewertung \(\nu\colon K \to \set Z \cup \{\infty\}\) von \(A\) surjektiv ist, existiert ein \(x \in \ideal m\)
		mit \(\nu(x) = 1\). Damit ist \(\ideal m = (x)\), und alle weiteren Ideale sind von der Form \(\ideal m_k = (x^k) = \ideal m^k\).
	\end{proof}
\end{frame}

\subsection{Charakterisierungen diskreter Bewertungsringe}

\begin{frame}{Charakterisierungen diskreter Bewertungsringe}
	\begin{proposition}<+->
		Sei \((A, \ideal m, F)\) ein eindimensionaler noetherscher lokaler Integritätsbereich.
		Dann sind folgende Aussagen äquivalent:
		\begin{enumerate}[<+->]
		\item<.->
			\(A\) ist ein diskreter Bewertungsring.
		\item
			\(A\) ist ganz abgeschlossen.
		\item
			\(\ideal m\) ist ein Hauptideal.
		\item
			\(\dim_F \ideal m/\ideal m^2 = 1\).
		\item
			Jedes (nicht verschwindende) Ideal von \(A\) ist eine Potenz von \(\ideal m\).
		\item
			Es existiert ein \(x \in A\), so daß jedes (nicht verschwindende) Ideal von der Form
			\((x^k)\) mit \(k \ge 0\) ist.
		\end{enumerate}
	\end{proposition}
\end{frame}

\begin{frame}{Beweis der Charakterisierungen}
	\begin{proof}[Beweis]<+->
		\renewcommand{\qedsymbol}{}
		\begin{enumerate}[<+->]
		\item<.->
			Bewertungsringe sind ganz abgeschlossen.
		\item
			Sei \(A\) ganz abgeschlossen. Sei \(a \in \ideal m \setminus \{0\}\). Dann ist \(\ideal a \coloneqq (a)\)
			ein \(\ideal m\)-primäres Ideal (da \(\ideal m\) das einzige nicht verschwindende
			Primideal ist), also \(\sqrt{\ideal (a)} = \ideal m\). Da \(A\) noethersch ist,
			existiert ein \(n\) mit \(\ideal m^n \subset \ideal a\). Wir wählen \(n\) minimal.
			\\
			Wähle ein \(b \in \ideal m^{n - 1}\) mit \(b \notin \ideal a\). Sei \(x \coloneqq \frac a b\).
			Damit ist \(x^{-1} \notin A\), also nicht ganz über \(A\). Damit ist \(x^{-1} \ideal m \subsetneq \ideal m\),
			denn sonst wäre \(\ideal m\) ein treuer \(A[x^{-1}]\)-Modul, der endlich erzeugt ist als \(A\)-Modul.
			\\
			Aber \(x^{-1} \ideal m \subset A\) nach Konstruktion, also \(x^{-1} \ideal m = A\), also \(\ideal m = Ax = (x)\).
			\\
			Damit ist \(\ideal m\) ein Hauptideal.
		\item
			Sei \(\ideal m\) ein Hauptideal. Dann ist \(\dim_F \ideal m/\ideal m^2 \leq 1\). Da \(\ideal m \neq \ideal m^2\),
			da \(\dim A > 0\) folgt \(\dim_F \ideal m/\ideal m^2 = 1\).
			\qedhere
		\end{enumerate}
	\end{proof}
\end{frame}

\begin{frame}{Fortsetzung des Beweises}
	\begin{proof}[Fortsetzung des Beweises]<+->
		\renewcommand{\qedsymbol}{}
		\begin{enumerate}[<+->]
		\item<.->
			Sei \(\dim_F \ideal m/\ideal m^2 = 1\). Sei \(\ideal a\) ein nicht verschwindendes Ideal von \(A\).
			Wir können uns auf den Fall \(\ideal a \neq (1)\) beschränken.
			Wir haben schon gesehen, daß \(\ideal m^n \subset \ideal a\) für ein \(n\). Es ist \(A/\ideal m^n\) ein
			lokaler artinscher Ring, dessen Zariskischer Kotangentialraum höchstens eindimensional ist.
			\\
			Damit ist \((A/\ideal m^n) \ideal a\) notwendigerweise eine Potenz von \((A/\ideal m^n) \ideal m\), wie wir schon
			gezeigt haben.
			\\
			Folglich ist \(\ideal a\) eine Potenz von \(\ideal m\).
		\item
			Sei jedes nicht verschwindende Ideal von \(A\) eine Potenz von \(\ideal m\). Da \(\dim A > 0\), ist \(\ideal m
			\neq \ideal m^2\), also existiert ein \(x \in \ideal m \setminus \ideal m^2\).
			\\
			Nach Voraussetzung ist \((x) = \ideal m^r\) für ein \(r \ge 0\), also \(r = 1\). Damit ist \((x) = \ideal m\) und
			\((x^k) = \ideal m^k\) für alle \(k \ge 0\).
			\\
			Damit sind alle nicht verschwindenden Ideale von der Form \((x^k)\).
			\qedhere
		\end{enumerate}
	\end{proof}
\end{frame}

\begin{frame}{Ende des Beweises}
	\begin{proof}[Ende des Beweises]<+->
		\begin{enumerate}
		\item<.->
			Seien alle nicht verschwindenden Ideale von der Form \((x^k)\) mit \(x \in A\). Da \(\ideal m\) maximal ist,
			folgt \(\ideal m = (x)\). Da \(\dim A > 0\), damit \((x^k) \neq (x^{k + 1})\). Ist also \(a \in A \setminus \{0\}\),
			so gilt \((a) = (x^k)\) für genau ein \(k \in \set N_0\). Durch die Setzung \(\nu(a) \coloneqq k\) wird eine
			diskrete Bewertung auf dem Quotientenkörper von \(A\) mit Bewertungsring \(A\) definiert.
			\\
			Folglich ist \(A\) ein diskreter Bewertungsring.
			\qedhere
		\end{enumerate}
	\end{proof}
\end{frame}

