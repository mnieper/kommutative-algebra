\subsection{Noethersche Ringe}

\begin{exercise}
	Sei \(A\) ein nicht noetherscher kommutativer Ring. Zeige, daß die
	Menge \(\mathfrak S\) der nicht endlich erzeugten Ideale von \(A\)
	ein maximales Element besitzt und daß maximale Elemente Primideale sind.
	
	(Tip: Sei \(\ideal a\) ein maximales Element von \(\mathfrak S\). Seien
	\(x, y \in A\) mit \(xy \in \ideal a\), aber \(x, y \notin \ideal a\).
	Zeige, daß ein endlich erzeugtes Ideal \(\ideal a_0 \subset \ideal a\) mit
	\(\ideal a_0 + (x) = \ideal a + (x)\) und \(\ideal a = \ideal a_0 +
	x \cdot (a : x)\) existiert. Da \(\ideal a \subsetneq (\ideal a : x)\), kann
	\(\ideal a\) nicht maximal sein. Widerspruch.)
	
	Folgere, daß ein kommutativer Ring, in dem jedes Primideal endlich erzeugt
	ist, ein noetherscher Ring ist.
\end{exercise}

\begin{exercise}
	\label{exer:nilp_powerseries}
	Sei \(A\) ein noetherscher kommutativer Ring. Zeige, daß eine Potenzreihe
	\(f = \sum\limits_{n = 0}^\infty a_n x^n \in \ps A x\) genau dann nilpotent ist,
	wenn alle \(a_n\) nilpotent sind.
\end{exercise}

\begin{exercise}
	Sei \(\ideal a\) ein irreduzibles Ideal in einem kommutativen Ring \(A\).
	Zeige, daß die folgenden Aussagen äquivalent sind:
	\begin{enumerate}
	\item
		Das Ideal \(\ideal a\) ist ein Primärideal.
	\item
		Für jede mutliplikativ abgeschlossene Teilmenge \(S\) von \(A\)
		existiert ein \(x \in S\) mit \(A \cap S^{-1} \ideal a = (\ideal a : x)\).
	\item
		Für alle \(x \in A\) ist die Kette
		\((\ideal a : x) \subset (\ideal a : x^2) \subset \dotsb\)
		stationär.
	\end{enumerate}
\end{exercise}

\begin{exercise}
	Sei \(A\) ein noetherscher kommutativer Ring. Seien weiter \(B\) eine
	endlich erzeugte kommutative \(A\)-Algebra und \(G\) eine endliche Gruppe
	von \(A\)-Algebrenautomorphismen von \(B\). Zeige, daß
	\(B^G \coloneqq \{y \in B \mid \forall g \in G\colon g(y) = y\}\) eine
	endlich erzeugte \(A\)-Algebra ist.
\end{exercise}

\begin{exercise}
	Sei \(K\) ein endlich erzeugter kommutativer Ring, das heißt \(K\) ist
	endlich erzeugt als \(\set Z\)-Algebra. Zeige: Ist \(K\) ein Körper, so
	ist \(K\) ein endlicher Körper.
	
	(Tip: Ist \(K\) von Charakteristik \(0\), so haben wir \(\set Z \subset
	\set Q \subset K\). Da \(K\) endlich über \(\set Z\) erzeugt ist, ist
	\(K\) dann auch endlich über \(\set Q\) erzeugt, und ist
	nach~\prettyref{cor:weak_hilbert1} damit ein endlich erzeugter
	\(\set Q\)-Vektorraum. Leite aus~\prettyref{prop:intermediate_ring} dann
	einen Widerspruch her.
	
	Also ist \(K\) von positiver Charakteristik \(p\), also endlich erzeugt
	als \(\set Z/(p)\)-Algebra. Nutze wieder~\prettyref{cor:weak_hilbert1}.)
\end{exercise}

\begin{exercise}
	Sei \(A\) ein kommutativer Ring, so daß \(A[x]\) noethersch ist. Ist
	dann auch \(A\) noethersch?
\end{exercise}

\begin{exercise}
	\label{exer:stalks_are_noetherian}
	Sei \(A\) ein kommutativer Ring. Seien die Halme \(A_{\ideal m}\) für
	alle maximalen Ideale \(\ideal m\) von \(A\) noethersch. Sei weiter für
	alle \(x \in A \setminus \{0\}\) die Menge der maximalen Ideale mit
	\(x \in \ideal m\) endlich. Zeige, daß \(A\) noethersch ist.
	
	(Tip: Sei \(\ideal a \neq (0)\) ein Ideal in \(A\). Seien \(\ideal m_1,
	\dotsc, \ideal m_r\) die maximalen Ideale, welche \(\ideal a\) umfassen.
	Wähle \(x_0 \in \ideal a \neq (0)\). Seien \(\ideal m_1, \dotsc,
	\ideal m_{r + s}\) diejenigen maximalen Ideale, welche \(x_0\) enthalten.
	Da \(\ideal m_{r + 1}, \dotsc, \ideal m_{r + s}\) das Ideal \(\ideal a\)
	nicht umfassen, existieren \(x_j \in \ideal a\) mit \(x_j \notin
	\ideal m_{r + j}\) für \(1 \leq j \leq s\). Da die \(A_{\ideal m_i}\)
	noethersch sind, sind die \(A_{\ideal m_i} \ideal a\) endlich erzeugt.
	Damit existieren \(x_{s + 1}, \dotsc, x_t \in \ideal a\), deren Bilder
	in \(A_{\ideal m_i}\) die \(A_{\ideal m_i} \ideal a\) für
	\(1 \leq i \leq r\) erzeugen. Sei \(\ideal a_0 = (x_0, \dotsc, x_t)\).
	Zeige, daß \(A_{\ideal m} \ideal a_0 = A_{\ideal m} \ideal a\) für alle
	maximalen Ideale \(\ideal m\) von \(A\) und folgere
	mit~\prettyref{prop:inj_is_local}, daß \(\ideal a_0 = \ideal a\).)
\end{exercise}

\begin{exercise}
	Sei \(A\) ein kommutativer Ring.
	Sei \(M\) ein noetherscher \(A\)-Modul. Zeige, daß \(M[x]\) (siehe
	\prettyref{exer:poly_over_mod}) ein noetherscher \(A[x]\)-Modul ist.
\end{exercise}

\begin{exercise}
	Sei \(A\) ein kommutativer Ring, so daß der Halm \(A_{\ideal p}\)
	für jedes Primideal ein noetherscher Ring ist. Ist dann \(A\)
	notwendigerweise auch noethersch?
\end{exercise}

\begin{exercise}
	Sei \(A\) ein kommutativer Ring. Sei \(M\) ein noetherscher \(A\)-Modul.
	Zeige, daß jeder Untermodul \(N\) von \(M\) eine Primärzerlegung in
	\(M\) besitzt (siehe~\prettyref{exer:primary_decomp_for_mod}).
	
	(Tip: Imitiere die Beweise von \prettyref{lem:lasker1} und
	\prettyref{lem:lasker2}.)
\end{exercise}

\begin{exercise}
	\label{exer:primes_to_mod}
	Sei \(A\) ein noetherscher kommutativer Ring. Sei \(M\) ein endlich
	erzeugter \(A\)-Modul. Zeige, daß für ein Primideal \(\ideal p\) von \(A\)
	die folgenden Aussagen äquivalent sind:
	\begin{enumerate}
	\item
		Das Primideal \(\ideal p\) ist zu \(0\) in \(M\) assoziiert.
	\item
		Es existiert ein \(x \in M\) mit \(\ann(x) = \ideal p\).
	\item
		Es existiert ein Untermodul \(N\) von \(M\) mit \(N \cong A/\ideal p\).
	\end{enumerate}
	
	Folgere, daß eine Kette von Untermoduln \(0 = M_0 \subset M_1 \dotsb
	M_r = M\) existiert, so daß jeder Quotient \(M_{i + 1}/M_i\) von der
	Form \(A/\ideal p_i\) mit einem Primideal \(\ideal p_i\) in \(A\) ist.
\end{exercise}

\begin{exercise}
	Sei \(A\) ein noetherscher kommutativer Ring. Seien
	\(\ideal a = \bigcap\limits_{i = 1}^r \ideal b_i
	= \bigcap\limits_{j = 1}^s \ideal c_j\) zwei minimale Zerlegungen
	eines Ideals \(\ideal a\) in irreduzible Ideale. Zeige, daß
	\(r = s\) und daß eine Permutation \(\sigma \in \SG_n\) mit
	\(\sqrt{\ideal b_i} = \sqrt{\ideal c_{\sigma(i)}}\) existiert.
	
	(Tip: Zeige, daß für alle \(1 \leq i \leq r\) ein \(j\) mit
	\(\ideal a = \ideal b_1 \cap \dotsb \cap \ideal b_{i - 1} \cap
	\ideal c_j \cap \ideal b_{i + 1} \cap \dotsb \ideal b_r\) existiert.)
	
	Formuliere und beweise eine entsprechende Aussage für Moduln.
\end{exercise}

\begin{exercise}
	\label{exer:groth_group}
	Sei \(A\) ein noetherscher kommutativer Ring. Mit \(\mathfrak F(A)\)
	bezeichnen wir die Menge der Isomorphieklassen \([M]\) endlich erzeugter
	\(A\)-Moduln \(M\). Sei \(C\) die durch \(\mathfrak F(A)\) frei erzeugte
	abelsche Gruppe, das heißt Elemente in \(C\) sind formale (endliche)
	\(\set Z\)-Linearkombinationen von Isomorphieklassen endlich erzeugter
	\(A\)-Moduln. Jeder exakten Sequenz \(0 \to M' \to M \to M'' \to 0\)
	endlich erzeugter \(A\)-Moduln ordnen wir das Element
	\([M] - [M'] - [M''] \in C\) zu. Sei \(D\) die von diesen Elementen für alle
	exakten Sequenzen erzeugte Untergruppe. Die Quotientengruppe
	\(\GrothK(A) \coloneqq C/D\) heißt die \emph{Grothendiecksche Gruppe
	von \(A\)}. Für jeden endlich erzeugten \(A\)-Modul \(A\) bezeichnen wir mit
	\(\gamma(M) = \gamma_A(M)\) das Bild von \([M]\) in \(\GrothK(A)\).
	Zeige:
	\begin{enumerate}
	\item
		Für jede additive Funktion \(\lambda\) mit Werten in einer abelschen
		Gruppe \(G\), welche auf der Klasse der endlich erzeugten \(A\)-Moduln
		definiert ist, existiert genau ein Gruppenhomorphismus
		\(\lambda_0\colon \GrothK(A) \to G\) mit \(\lambda(M) = 
		\lambda_0(\gamma(M))\) für alle endlich erzeugten \(A\)-Moduln \(M\).
	\item
		Zeige, daß \(\GrothK(A)\) von Elementen der Form \(\gamma(A/\ideal p)\)
		erzeugt wird, wobei \(\ideal p\) ein Primideal in \(a\) ist.
		
		(Tip: \prettyref{exer:primes_to_mod}.)
	\item
		Zeige, daß \(\GrothK(A) \cong \set Z\), wenn \(A\) ein Körper oder
		allgemeiner ein Hauptidealbereich ist.
	\item
		Sei \(f^*\colon A \to B\) ein endlicher Homomorphismus
		noetherscher kommutativer Ringe. Zeige, daß ein Gruppenhomomorphismus
		\(f_!\colon \GrothK(B) \to \GrothK(A)\) mit \(f_!(\gamma_B(N))
		= \gamma_A(N^A)\) für alle endlich erzeugten \(B\)-Moduln
		\(N\) existiert.
	
		Sei \(g^*\colon B \to C\) ein weiterer Homomorphismus noetherscher
		kommutativer Ringe. Zeige, daß \((f \circ g)_! = f_! \circ g_!\colon
		\GrothK(C) \to \GrothK(A).\).
	\end{enumerate}
\end{exercise}

