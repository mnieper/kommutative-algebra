\section{Erster Cohen--Seidenbergscher Satz}

\subsection{Körpererweiterungen}

\begin{frame}{Körper in einer ganzen Erweiterung}
	\begin{proposition}<+->
		\label{prop:fields_and_integral_extensions}
		Sei \(A \subset B\) eine ganze Erweiterung von Integritätsbereichen.
		Dann ist \(B\) genau dann ein Körper, wenn \(A\) ein Körper ist.
	\end{proposition}
	\begin{proof}<+->
		\begin{enumerate}[<+->]
		\item<.->
			Sei \(A\) ein Körper. Sei \(y \in B\) mit \(y \neq 0\). Sei
			\(y^n + a_1 y^{n - 1} + \dotsb + a_n = 0\) mit \(a_i \in A\) eine
			Ganzheitsbedingung minimalen Grades. Da \(B\) ein
			Integritätsbereich ist, ist \(a_n \neq 0\), also
			\(y^{-1} = -a_n^{-1} (y^{n - 1} + a_1 y^{n - 2} + \dotsb
			+ a_{n - 1}) \in B\). Damit ist \(B\) ein Körper.
		\item
			Sei umgekehrt \(B\) ein Körper. Sei \(x \in A\) mit \(x \neq 0\).
			Dann ist \(x^{-1} \in B\), also ganz über \(A\), so daß eine
			Ganzheitsbedingung \(x^{-n} + a_1 x^{-n + 1} + \dotsb + a_n = 0\)
			mit \(a_i \in A\) existiert. Es folgt, daß
			\(x^{-1} = -(a_1 + a_2 x + \dotsb + a_n x^{n - 1}) \in A\). Damit
			ist \(A\) ein Körper. 
			\qedhere
		\end{enumerate}
	\end{proof}
\end{frame}

\begin{frame}{Maximale Ideale in einer ganzen Erweiterung}
	\begin{corollary}<+->
		Sei \(A \subset B\) eine ganze Erweiterung kommutativer Ringe. Sei
		\(\ideal q\) ein Primideal in \(B\) und \(\ideal p \coloneqq A \cap
		\ideal q\). Dann ist \(\ideal q\) genau dann ein maximales Ideal, wenn
		\(\ideal p\) ein maximales Ideal ist.
	\end{corollary}
	\begin{proof}<+->
		Es ist \(A/\ideal p \subset B/\ideal q\) eine ganze Erweiterung
		von Integritätsbereichen. Es ist \(A/\ideal p\) genau dann ein Körper,
		wenn \(\ideal p\) maximal ist. Ebenso ist \(B/\ideal q\) genau dann
		ein Körper, wenn \(\ideal q\) maximal ist.
	\end{proof}
\end{frame}

\subsection{Primideale in ganzen Erweiterungen}

\begin{frame}{Primideale in einer ganzen Erweiterung}
	\begin{corollary}<+->
		\label{cor:equality_of_primes_in_integral_extension}
		Sei \(A \subset B\) eine ganze Erweiterung kommutativer Ringe. Seien
		\(\ideal q \subset \ideal q'\) zwei Primideale in \(B\) mit
		\(\ideal p \coloneqq A \cap \ideal q = A \cap \ideal q'\). Dann gilt
		\(\ideal q = \ideal q'\).
	\end{corollary}
	\begin{proof}<+->
		\begin{enumerate}[<+->]
		\item<.->
			Es ist \(B_{\ideal p}\) ganz über \(A_{\ideal p}\).
			Seien \(\ideal m \coloneqq A_{\ideal p} \ideal p\), \(\ideal n
			\coloneqq B_{\ideal p} \ideal q\) und
			\(\ideal n' \coloneqq B_{\ideal p} \ideal q'\).
		\item
			Dann ist \(\ideal m\) das maximale Ideal in \(A_{\ideal p}\).
			Weiter gilt \(\ideal n \subset \ideal n'\) und
			\(A_{\ideal p} \cap \ideal n = A_{\ideal p} \cap \ideal n' =
			\ideal m\).
		\item
			Aus der Maximalität von \(\ideal m\) folgt, daß \(\ideal n\)
			maximal ist, also \(\ideal n = \ideal n'\). Es folgt
			\(\ideal q = \ideal q'\).
			\qedhere
		\end{enumerate}
	\end{proof}
\end{frame}

\begin{frame}{Existenz von Primidealen in ganzen Ringerweiterungen}
	\begin{theorem}<+->
		\label{thm:existence_of_primes_in_integral_extensions}
		Sei \(A \subset B\) eine ganze Erweiterung kommutativer Ringe.
		Sei \(\ideal p\) ein Primideal von \(A\). Dann existiert ein
		Primideal \(\ideal q\) von \(B\) mit \(A \cap \ideal q = \ideal p\).
	\end{theorem}
	\begin{proof}<+->
		\begin{enumerate}[<+->]
		\item<.->
			Sei \(\ideal n\) ein maximales Ideal von \(B_{\ideal p}\). Da
			\(B_{\ideal p}\) ganz über \(A_{\ideal p}\) ist, ist
			\(\ideal m \coloneqq A_{\ideal p} \cap \ideal n\) ein maximales
			Ideal von \(A_{\ideal p}\) und damit das einzige maximale Ideal
			in \(A_{\ideal p}\).
		\item
			Sei \(\ideal q \coloneqq B \cap \ideal n\). Dann ist \(\ideal q\)
			ein Primideal in \(B\). Weiter ist
			\(A \cap \ideal q = A \cap (A_{\ideal p} \cap \ideal n)
			= A \cap \ideal m = \ideal p\).
			\qedhere
		\end{enumerate}
	\end{proof}
\end{frame}

\begin{frame}{Der erste Cohen--Seidenbergsche Satz}
	\begin{theorem}["`Going-up"']<+->
		Sei \(A \subset B\) eine ganze Erweiterung kommutativer Ringe. Sei
		\(\ideal p_1 \subset \dotsb \subset \ideal p_n\) eine Kette von
		Primidealen in \(A\) und
		\(\ideal q\colon \ideal q_1 \subset \dotsb \subset \ideal q_m\), \(m \le n\), eine
		Kette von Primidealen in \(B\) mit \(\ideal p_i = A \cap \ideal q_i\)
		für \(i \le m\). Dann kann die Kette \(\ideal q\) zu einer Kette
		\(\ideal q_1 \subset \dotsb \subset \ideal q_n\) mit \(A \cap
		\ideal q_i = \ideal p_i\) erweitert werden.
	\end{theorem}
	\begin{proof}<+->
		\begin{enumerate}[<+->]
		\item<.->
			Es reicht, den Fall \(m = 1\), \(n = 2\) zu behandeln. Seien \(\bar A
			\coloneqq A/\ideal p_1\) und \(\bar B \coloneqq B/\ideal q_1\).
			Dann ist \(\bar B\) ganz über \(\bar A\).
		\item
			 Damit existiert ein Primideal \(\bar{\ideal q}_2\) von \(\bar B\)
			 mit \(\bar A \cap \bar{\ideal q}_2 = \bar{\ideal p}_2
			 \coloneqq \bar A \ideal p_2\).
		\item
			Damit ist \(\ideal q_2 \coloneqq B \cap \bar{\ideal q}_2\) ein
			Primideal von \(B\) mit den gewünschten Eigenschaften.
			\qedhere
		\end{enumerate}
	\end{proof}
\end{frame}

