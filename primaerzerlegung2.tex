\section{Primärzerlegung II}

\subsection{Primärzerlegung und das Nullideal} 

\begin{frame}{Vereinigung der zu einem Ideal assoziierten Primideale}
	 \begin{proposition}<+->
	 	\label{prop:union_of_assoc_primes}
		Sei \(\ideal a\) ein zerlegbares Ideal eines kommutativen Ringes \(A\). Sei \(\ideal a = \bigcap\limits_{i = 1}^n
		\ideal q_i\) eine minimale Primärzerlegung. Sei \(\ideal p_i \coloneqq \sqrt{\ideal q_i}\). Dann gilt
		\(\bigcup\limits_{i = 1}^n \ideal p_i = \{x \in A \mid (\ideal a : x) \neq \ideal a\}\).
	\end{proposition}
	\begin{corollary}<+->
		Ist das Nullideal in \(A\) zerlegbar, so ist die Menge \(D\) der Nullteiler von \(A\) die Vereinigung der zu \((0)\)
		assoziierten Primideale.
	\end{corollary}
	\begin{proof}[Beweis der Proposition]<+->
		Aus der Zerlegbarkeit von \(\ideal a\) folgt die Zerlegbarkeit von \((0) \subset A/\ideal a\), denn \((0) = \bigcap_{i \in I}
		\bar{\ideal q}_i\), wobei \(\bar{\ideal q}_i\) das Bild von \(\ideal q_i\) in \(A/\ideal a\) ist, welches primär ist.
		\\
		Damit reicht es, die Folgerung zu beweisen.
	\end{proof}
\end{frame}

\begin{frame}{Beweis der Folgerung}
	\begin{proof}[Beweis der Folgerung]<+->
		\begin{enumerate}[<+->]
		\item<.->
			Zunächst ist \(D = \bigcup\limits_{x \neq 0} \sqrt{(0 : x)}\).
			\\
			Auf der anderen Seite wissen wir aus dem Beweis des ersten Eindeutigkeitssatzes, daß
			\(\sqrt{(0 : x)} = \bigcap\limits_{\ideal q_j \not\ni x} \ideal p_j \subset \ideal p_i\) für ein \(i\). Damit
			gilt \(D \subset \bigcup\limits_{i = 1}^n \ideal p_i\).
		\item
			Auf der anderen Seite ist jedes \(\ideal p_i\) von der Form \(\sqrt{(0 : x)}\) mit \(x \neq 0\), also
			\(\bigcup\limits_{i = 1}^n \ideal p_i \subset D\).
			\qedhere
		\end{enumerate}
	\end{proof}
\end{frame}

\begin{frame}{Die Menge der Nullteiler und die Menge der nilpotenten Elemente}
	\begin{remark}<+->
		Für einen kommutativen Ring \(A\), in dem das Nullideal zerlegbar ist, gilt also:
		\begin{enumerate}[<+->]
		\item<.->
			Die Menge der Nullteiler von \(A\) ist die Vereinigung aller Primideale, die assoziiert zu \((0)\) sind.
		\item
			Die Menge der nilpotenten Elemente ist der Schnitt aller (isolierten) Primideale, die zu \((0)\) assoziiert sind.
		\end{enumerate}
	\end{remark}
\end{frame}

\subsection{Primäre Ideale und Lokalisierung}

\begin{frame}{Primäre Ideale in Lokalisierungen}
	\begin{proposition}<+->
		\label{prop:primaries_in_localisation}
		Sei \(\ideal p\) ein Primideal in einem kommutativen Ring \(A\). Sei \(\ideal q\) ein \(\ideal p\)-primäres Ideal.
		Sei \(S \subset A\) multiplikativ abgeschlossen. Dann gilt:
		\begin{enumerate}[<+->]
		\item<.->
			Ist \(S \cap \ideal p \neq \emptyset\), so folgt \(S^{-1} \ideal q = (1)\).
		\item
			Ist \(S \cap \ideal p = \emptyset\), so ist \(S^{-1} \ideal q\) ein \(S^{-1} \ideal p\)-primäres Ideal
			und für seine Kontraktion gilt \(A \cap S^{-1} \ideal q = \ideal q\).
		\end{enumerate}
	\end{proposition}
	\begin{proof}<+->
		\begin{enumerate}[<+->]
		\item<.->
			Ist \(s \in S \cap \ideal p\), so ist \(s^n \in S \cap \ideal q\) für ein \(n \in \set N_0\). Damit enthält
			\(S^{-1} \ideal q\) das Element \(\frac{s^n} 1\), welches eine Einheit in \(S^{-1} A\) ist.
		\item
			Ist \(S \cap \ideal p = \emptyset\), so folgt aus \(s \in S\) und \(as \in \ideal q\) schon \(a \in \ideal q\).
			Wegen \(A \cap S^{-1} \ideal q = \bigcup\limits_{s \in S} (\ideal q : s)\) folgt damit \(A \cap S^{-1} \ideal q
			= \ideal q\).
			Außerdem ist \(\sqrt{S^{-1} \ideal q} = S^{-1} \sqrt{\ideal q} = S^{-1} \ideal p\). Daß \(S^{-1} \ideal q\) primär ist,
			ist schließlich einfach.
			\qedhere
		\end{enumerate}
	\end{proof}
\end{frame}

\begin{frame}{Korrespondenz zwischen primären Idealen in einer Lokalisierung}
	\begin{corollary}<+->
		\label{cor:correspondence_for_primaries}
		Sei \(A\) ein kommutativer Ring. Sei \(S \subset A\) multiplikativ abgeschlossen.
		Durch \(\ideal r = S^{-1} \ideal q\) ist eine bijektive, ordnungserhaltende Korrespondenz
		zwischen den primären Idealen \(\ideal q\) von \(A\) mit \(\sqrt{\ideal q} \cap S = \emptyset\) und den
		primären Idealen von \(S^{-1} A\) gegeben. 
	\end{corollary}
	\begin{proof}<+->
		Folgt aus der letzten Proposition zusammen mit der Tatsache, daß die Kontraktion eines primären Ideals
		ein primäres Ideal ist.
	\end{proof}
\end{frame}

\subsection{Sättigung eines Ideals}

\begin{frame}{Definition der Sättigung eines Ideals}
	\begin{definition}<+->
		Sei \(A\) ein kommutativer Ring. Sei \(S \subset A\) multiplikativ
		abgeschlossen. Die \emph{Sättigung \(S(\ideal a)\) eines Ideals
		\(\ideal a\) von \(A\) bezüglich \(S\)} ist das Ideal
		\(S(\ideal a) \coloneqq A \cap S^{-1} \ideal a\), also die Kontraktion
		der Erweiterung des Ideals bezüglich der Lokalisierung \(S^{-1} A\).
	\end{definition}
\end{frame}

\begin{frame}{Sättigung zerlegbarer Ideale}
	\begin{proposition}<+->
		\label{prop:sat_decomp}
		Sei \(\ideal a\) ein zerlegbares Ideal in einem kommutativen Ring \(A\).
		Sei \(S \subset A\) multiplikativ abgeschlossen. Sei
		\(\ideal a = \ideal q_1 \cap \dotsb \cap \ideal q_n\) eine minimale
		Primärzerlegung von \(\ideal a\). Sei \(\ideal p_i \coloneqq
		\sqrt{\ideal q_i}\). Weiter seien die \(\ideal q_i\) so
		numeriert, daß \(S \cap \ideal p_i = \emptyset \iff
		i \leq m\) für ein \(0 \leq m \leq n\). Dann sind
		\(S^{-1} \ideal a = \bigcap\limits_{i = 1}^m S^{-1} \ideal q_i\)
		und \(S(\ideal a) = \bigcap\limits_{i = 1}^m \ideal q_i\)
		minimale Primärzerlegungen.
	\end{proposition}
	\begin{proof}<+->
		\begin{enumerate}[<+->]
		\item<.->
			\(S^{-1} \ideal a = \bigcap\limits_{i = 1}^n S^{-1} \ideal q_i
			= \bigcap\limits_{i = 1}^m S^{-1} \ideal q_i\), und
			\(S^{-1} \ideal q_i\) ist ein
			\(S^{-1} \ideal p_i\)-primäres Ideal für \(i \leq m\).
		\item
			Da die \(\ideal p_i\) paarweise verschieden sind, gilt dies auch
			für die \(S^{-1} \ideal p_i\) für \(i \leq m\), so daß
			\(\bigcap\limits_{i = 1}^m S^{-1} \ideal q_i\) minimal ist.
		\item
			Kontraktion liefert die entsprechende Aussage
			für \(S(\ideal a)\).
			\qedhere
		\end{enumerate}
	\end{proof}
\end{frame}

\begin{frame}{Isolierte Mengen assoziierter Primideale}
	Sei \(\ideal a\) ein zerlegbares Ideal in einem kommutativen Ring \(A\).
	\begin{definition}<+->
		Eine Menge \(\mathfrak S\) von an \(\ideal a\) assoziierten Primidealen
		heißt \emph{isoliert}, falls sie folgende Bedingung erfüllt: Ist
		\(\ideal p'\) ein zu \(\ideal a\) assoziiertes Primideal und ist
		\(\ideal p' \subset \ideal p\) für ein \(\ideal p \in \mathfrak S\),
		so folgt \(\ideal p' \in \mathfrak S\).
	\end{definition}
	\begin{proposition}<+->
		Sei \(\mathfrak S\) eine isolierte Menge von an \(\ideal a\)
		assoziierten Primidealen. Dann ist \(S \coloneqq A \setminus
		\bigcup\limits_{\ideal p \in \mathfrak S} \ideal p\) multiplikativ
		abgeschlossen, und für jedes an \(\ideal a\) assoziierte Primideal
		\(\ideal p'\) gilt:
		\begin{enumerate}[<+->]
		\item<.->
			Ist \(\ideal p' \in \mathfrak S\), so gilt
			\(\ideal p' \cap S = \emptyset\).
		\item
			Ist \(\ideal p' \notin \mathfrak S\), so ist
			\(\ideal p' \cap S \neq \emptyset\).
			\qed
		\end{enumerate}
	\end{proposition}
\end{frame}

\begin{frame}{Der zweite Eindeutigkeitssatz}
	\begin{theorem}[Der zweite Eindeutigkeitssatz]<+->
		\label{thm:second_uniqueness}
		Sei \(\ideal a\) ein zerlegbares Ideal in einem kommutativen
		Ring \(A\). Sei \(\ideal a = \bigcap\limits_{i = 1}^n \ideal q_i\)
		eine minimale Primärzerlegung. Sei \(\ideal p_i \coloneqq
		\sqrt{\ideal q_i}\). Ist dann \(\{\ideal p_1, \dots, \ideal p_m\}\)
		eine isolierte Menge von an \(\ideal a\) assoziierten Primidealen,
		so ist \(\bigcap\limits_{i = 1}^m \ideal q_i\) unabhängig von der
		Zerlegung.
	\end{theorem}
	\begin{proof}<+->
		Sei \(S \coloneqq A \setminus \bigcup\limits_{i = 1}^m \ideal p_i\).
		Dann ist \(S(\ideal a) = \bigcap\limits_{i = 1}^m \ideal q_i\), also
		unabhängig von der Zerlegung, da die \(\ideal p_i\) nur von
		\(\ideal a\) abhängen.
	\end{proof}
\end{frame}

\begin{frame}{Eindeutigkeit der isolierten primären Komponenten}
	\begin{corollary}<+->
		\label{cor:second_uniqueness}
		Sei \(\ideal a\) ein zerlegbares Ideal in einem kommutativen
		Ring \(A\). Sei \(\ideal a = \bigcap\limits_{i = 1}^n \ideal q_i\)
		eine minimale Primärzerlegung. Sei \(\ideal p_i \coloneqq
		\sqrt{\ideal q_i}\). Dann sind die \(\ideal q_i\), für die \(\ideal p_i\)
		ein isoliertes Primideal von \(\ideal a\) ist, die \alert{isolierten
		primären Komponenten von \(\ideal a\)}, eindeutig bestimmt.
	\end{corollary}
	\begin{remark}<+->
		Die eingebetteten primären Komponenten sind dagegen im allgemeinen nicht
		eindeutig durch das Ideal bestimmt.
		\\
		Später werden wir Beispiele sehen, in denen es unendlich viele
		Möglichkeiten für jede eingebettete Komponente gibt.
	\end{remark}
\end{frame}

