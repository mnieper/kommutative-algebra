\section{Bewertungsringe}

\subsection{Definition und erste Eigenschaften von Bewertungsringen}

\begin{frame}{Definition eines Bewertungsringes}
	\begin{definition}<+->
		Sei \(B\) ein Integritätsbereich mit Quotientenkörper \(K\). Dann heißt
		\(B\) ein \emph{Bewertungsring für \(K\)}, falls 
		\(x \in B\) oder \(x^{-1} \in B\) (oder beides) für alle \(x \in K^\units\).
	\end{definition}
	\begin{example}<+->
		Seien \(K\) ein Körper und \(B\) ein Bewertungsring für \(K\). Ist
		dann \(B'\) ein Unterring von \(K\) mit \(B \subset B'\), so ist
		\(B'\) ein Bewertungsring für \(K\).
	\end{example}
\end{frame}

\begin{frame}{Bewertungsringe sind lokale Ringe}
	\begin{proposition}<+->
		Sei \(B\) ein Bewertungsring. Dann ist \(B\) ein lokaler Ring.
	\end{proposition}
	\begin{proof}<+->
		\begin{enumerate}[<+->]
		\item<.->
			Sei \(\ideal m \coloneqq B \setminus B^\units\). Ist \(K\) der
			Quotientenkörper von \(B\), so gilt damit für \(x \in K\), daß
			\(x \in \ideal m\) genau dann, wenn \(x = 0\) oder \(x^{-1} \notin B\).
		\item
			Ist \(a \in B\) und \(x \in \ideal m\), so ist \(ax \in \ideal m\),
			da ansonsten \((ax)^{-1} \in B\) und damit \(x^{-1} = a (ax)^{-1}
			\in B\).
		\item
			Sind \(x, y \in \ideal m\), so gilt \(xy^{-1} \in B\) oder
			\(x^{-1} y \in B\). Ohne Einschränkung sei \(xy^{-1} \in B\),
			also \(x + y = (1 + xy^{-1}) y \in B \ideal m \subset \ideal m\).
		\item
			Damit ist \(\ideal m\) ein Ideal in \(B\), und \(B\) daher ein
			lokaler Ring.
			\qedhere
		\end{enumerate}
	\end{proof}
\end{frame}

\begin{frame}{Ganze Abgeschlossenheit von Bewertungsringen}
	\begin{proposition}<+->
		Sei \(B\) ein Bewertungsring. Dann ist \(B\) ganz abgeschlossen.
	\end{proposition}
	\begin{proof}<+->
		Sei \(x \in K\) ein ganzes Element im Quotientenkörper \(K\) von \(B\),
		also \(x^n + b_1 x^{n - 1} + \dotsb + b_n = 0\) für gewisse \(b_i \in
		B\). Falls \(x \in B\), ist nichts zu zeigen. Ansonsten ist \(x^{-1}
		\in B\) und damit
		\(x = - (b_1 + b_2 x^{-1} + \dotsb + b_n x^{1 - n}) \in B\).
	\end{proof}
\end{frame}

\subsection{Existenz von Bewertungsringen}

\begin{frame}{Nicht fortsetzbare Homomorphismen in algebraisch abgeschlossene
	Körper}
	\begin{visibleenv}<+->
		Seien \(K\) ein Körper und \(L\) ein algebraisch abgeschlossener Körper.
		Wir wollen einen
		Ringhomomorphismus \(\phi\colon A \to L\) von einem Unterring \(A \subset K\)
		\emph{nicht fortsetzbar in \(K\)}
		nennen, falls für Unterringe \(A'\) von \(K\) mit \(A \subset
		A'\) und einem Ringhomomorphismus \(\phi'\colon A' \to L\) mit
		\(\phi'|_A = \phi\) schon \(A = A'\) gilt.
	\end{visibleenv}
	\begin{remark}<+->
		Aus dem Zornschen Lemma folgt, daß zu jedem Ringhomomorphismus
		\(\phi\colon A \to L\) ein Unterring \(B\) von \(K\) mit
		\(A \subset B\) und ein Ringhomomorphismus \(\psi\colon B \to K\)
		mit \(\psi|_A = \phi\) existiert, so daß \(\psi\) nicht fortsetzbar
		ist.
	\end{remark}
\end{frame}

\begin{frame}{Nicht fortsetzbare Homomorphismen sind auf lokalen Ringe definiert}
	\begin{lemma}<+->
		Seien \(K\) ein Körper und \(L\) ein algebraisch abgeschlossener Körper. Sei \(B
		\subset K\) ein	Unterring. Ist dann \(\psi\colon B \to L\) ein nicht
		in \(K\) fortsetzbarer Ringhomomorphismus, so ist \(B\) ein lokaler Ring mit
		maximalem Ideal \(\ideal m = \ker \psi\).
	\end{lemma}
	\begin{proof}<+->
		\begin{enumerate}[<+->]
		\item<.->
			Da \(\psi(B) \cong B/\ker \psi\) als Unterring der Körpers \(L\) ein
			Integritätsbereich ist, ist \(\ker \psi\) ein Primideal.
		\item
			Nach der universellen Eigenschaft der Lokalisierung, faktorisiert
			\(\psi\) über einen Homomorphismus \(\psi_{\ideal m}\colon
			B_{\ideal m} \to L\).
		\item
			Da \(\psi\colon B \to L\) nicht fortsetzbar ist, folgt
			\(B = B_{\ideal m}\), also ist \(B\) ein lokaler Ring mit maximalem
			Ideal \(\ideal m\).
			\qedhere
		\end{enumerate}
	\end{proof}
\end{frame}

\begin{frame}{Ein Hilfssatz über nicht fortsetzbare Homomorphismen}
	\begin{lemma}<+->
		Seien \(K\) ein Körper und \(L\) ein algebraisch abgechlossener
		Körper. Sei \(B \subset K\) ein Unterring.
		Sei \(\psi\colon B \to L\) ein nicht in \(K\) fortsetzbarer Ringhomomorphismus.
		Sei \(\ideal m\) das maximale Ideal von \(B\). Sei weiter
		\(x \in K^\units\). Dann gilt
		\(\ideal m[x] \neq B[x]\) oder \(\ideal m[x^{-1}] \neq B[x^{-1}]\).
	\end{lemma}
	\begin{proof}<+->
		\begin{enumerate}[<+->]
		\item<.->
			Angenommen, \(\ideal m[x] = B[x]\) und \(\ideal m[x^{-1}]
			= B[x^{-1}]\). Dann gibt es Gleichungen
			\(u_0 + u_1 x + \dotsb + u_m x^m = 1\) und
			\(v_0 + v_1 x^{-1} + \dotsb + v_n x^{-n} = 1\) mit \(u_i, v_j \in
			\ideal m\). Seien weiter \(m, n\) minimal gewählt und ohne
			Einschränkung \(m \ge n\).
		\item
			Aus \((1 - v_0) x^n = v_1 x^{n - 1} + \dotsb + v_n\) und
			\((1 - v_0) \in B^\units\) folgt
			\(x^n = w_1 x^{n - 1} + \dotsb + w_n\)
			für gewisse \(w_j \in \ideal m\).
		\item
			Indem wir \(x^m = w_1 x^{m - 1} + \dotsb + w_n x^{m - n}\) in
			\(u_0 + u_1 x + \dotsb + u_m x^m = 1\) substituieren, erhalten wir,
			daß \(m\) nicht minimal gewählt ist. Widerspruch.
			\qedhere
		\end{enumerate}
	\end{proof}
\end{frame}

\begin{frame}{Nicht fortsetzbare Homomorphismen sind auf Bewertungsringen
	definiert}
	\begin{theorem}<+->
		\label{thm:existence_of_valuation_rings}
		Seien \(K\) ein Körper und \(L\) ein algebraisch abgeschlossener Körper.
		Sei \(B \subset K\) ein Unterring. Ist dann \(\psi\colon B \to L\)
		ein nicht in \(K\) fortsetzbarer Ringhomomorphismus, so ist \(B\) ein
		Bewertungsring für \(K\).
	\end{theorem}
\end{frame}

\begin{frame}{Beweis des Satzes über die Existenz von Bewertungsringen}
	\begin{proof}<+->
		\begin{enumerate}[<+->]
		\item<.->
			Mit \(\ideal m\) bezeichnen wir das maximale Ideal von \(B\).
			Sei \(x \in K^\units\). Wir haben \(x \in B\) oder \(x^{-1} \in B\)
			zu zeigen. Nach dem letzten Hilfssatz können wir ohne Einschränkung
			annehmen, daß \(\ideal m[x] \neq (1) = B' \coloneqq B[x]\).
		\item
			Damit ist \(\ideal m[x]\) in einem maximalen Ideal \(\ideal m'\) von
			\(B'\) enthalten. Da \(\ideal m' \cap B \subsetneq B\)
			folgt \(\ideal m' \cap B = \ideal m\).
		\item
			Die Einbettung \(B \to B'\) induziert damit eine Körpereinbettung
			\(k \coloneqq B/\ideal m \to k' \coloneqq B'/\ideal m'\).
			Sei \(\bar x\) das Bild von \(x\) in \(k'\). Damit
			\(k' = k[\bar x]\). Damit ist \(k[\bar x]\) ein Körper, \(\bar x\) also
			algebraisch über \(k\) und \(k'\) damit eine endlich algebraische
			Körpererweiterung von \(k\).
		\item
			Da \(\ideal m = \ker \psi\), induziert \(\psi\) einen Homomorphismus
			\(\bar\psi\colon k \to L\) von Körpern. Da \(L\) algebraisch
			abgeschlossen ist, können wir diesen zu einem Homomorphismus
			\(\bar\psi'\colon k' \to L\) fortsetzen. Verknüpfung mit
			\(B' \to k'\) liefert eine Fortsetzung von \(\psi\). Also \(B = B'\)
			und damit \(x \in B\).
			\qedhere
		\end{enumerate}
	\end{proof}
\end{frame}

\begin{frame}{Ganzer Abschluß als Schnitt über Bewertungsringe}
	\begin{corollary}<+->
		Sei \(A \subset K\) ein Unterring eines Körpers. Dann ist
		der ganze Abschluß \(\bar A\) von \(A\) in \(K\) der Schnitt aller
		Bewertungsringe \(B\) von \(K\) mit \(B \supset A\).
	\end{corollary}
	\begin{proof}<+->
		\begin{enumerate}[<+->]
		\item<.->
			Sei \(B\) ein Bewertungsring von \(K\) mit \(B \supset A\). Da
			\(B\) ganz abgeschlossen ist, folgt \(B \supset \bar A\).
		\item
			Sei umgekehrt \(x \notin \bar A\). Dann ist \(x \notin A' \coloneqq
			A[x^{-1}] \subset K\). Damit ist \(x^{-1}\) keine Einheit in \(A'\),
			also in einem maximalen Ideal \(\ideal m\) von \(A'\) enthalten.
		\item
			Sei \(L\) ein algebraischer Abschluß des Körpers \(k' \coloneqq
			A'/\ideal m'\). Der Homomorphismus
			\(\phi\colon A \injto A' \surjto k' \injto L\) kann zu einem
			Bewertungsring \(B\) von \(K\) fortgesetzt werden. Da
			\(\phi(x^{-1}) = 0\) ist folglich \(x \notin B\).
			\qedhere
		\end{enumerate}
	\end{proof}
\end{frame}

\begin{frame}{Fortsetzbarkeit in endlich erzeugten Erweiterungen}
	\begin{proposition}<+->
		Sei \(A \subset B\) eine endlich erzeugte Erweiterung von
		Integritätsbereichen. Sei \(v \in B \setminus \{0\}\). Dann existiert
		ein \(u \in A \setminus \{0\}\) mit folgender Eigenschaft: Jeder
		Homomorphismus \(\phi\colon A \to L\) in einen algebraisch
		abgeschlossenen Körper mit \(\phi(u) \neq 0\) kann zu einem
		Homomorphismus \(\psi\colon B \to L\) mit \(\phi(v) \neq 0\)
		fortgesetzt werden.
	\end{proposition}
	\begin{proof}<+->
		\renewcommand\qedsymbol{}
		Mit Induktion über die Anzahl der Erzeuger von \(B\) über \(A\)
		können wir davon ausgehen, daß \(B\) von einem Element \(x\)
		erzeugt wird.
		\\
		Dann können zwei Fälle vorliegen: Entweder ist \(x\) über \(A\)
		transzendent, das heißt kein nicht
		verschwindendes Polynom über \(A\) besitzt \(x\) als Nullstelle.
		\\
		Oder \(x\) ist über \(A\) algebraisch, das heißt es gibt eine nicht
		triviale polynomielle Gleichung für \(x\) über dem Quotientenkörper von \(A\).
	\end{proof}
\end{frame}

\begin{frame}{Beweis für den transzendenten Fall}
	\begin{proof}[Transzendenter Fall]<+->
		\begin{enumerate}[<+->]
		\item<.->
			Sei \(x\) transzendent über \(A\).	
		\item
			Es existieren \(a_i \in A\) mit \(v = a_0 x^n + a_1 x^{n - 1} +
			\dotsb + a_n\). Wähle \(u = a_0\).
		\item
			Ist \(\phi\colon A \to L\) ein Homomorphismus mit
			\(\phi(u) \neq 0\), so existiert
			ein \(y \in L\) mit \(\phi(a_0) y^n + \phi(a_1) y^{n - 1} + \dotsb +
			\phi(a_n) \neq 0\), da \(L\) unendlich viele Elemente besitzt.
		\item
			Definiere \(\psi\colon B \to L\) durch \(\psi(x) = y\).
			\renewcommand\qedsymbol{}
			\qedhere
		\end{enumerate}
	\end{proof}
\end{frame}

\begin{frame}{Beweis für den algebraischen Fall}
	\begin{proof}[Algebraischer Fall]<+->
		\begin{enumerate}[<+->]
		\item<.->
			Sei \(x\) algebraisch über \(A\). Da \(v\) ein Polynom in \(x\)
			ist, ist damit auch \(v^{-1}\) algebraisch über \(x\). Damit
			existieren Gleichungen \(a_0 x^m + a_1 x^{m - 1} + \dotsb
			+ a_m = 0\) und \(a_0' v^{-n} + a_1' v^{1 - n} + \dotsb
			+ a_n' = 0\) mit \(a_i, a_j' \in A\) und \(u \coloneqq a_0 a_0'
			\neq 0\). 
		\item
			Sei \(\phi\colon A \to L\) mit \(\phi(u) \neq 0\). Dann kann
			\(\phi\) zunächst zu einem Homomorphismus \(A[u^{-1}] \to L\)
			fortgesetzt werden und danach zu einem Homomorphismus \(\chi\colon
			C \to L\), wobei \(C\) ein Bewertungsring mit
			\(C \supset A[u^{-1}]\) ist. 
		\item
			Aus der ersten der obigen Gleichungen folgt, daß \(x\) ganz über
			\(A[u^{-1}]\) ist, daß also \(x \in C\), so daß sogar \(C \supset
			B\), insbesondere also \(v \in C\).
		\item
			Analog ist \(v^{-1} \in C\), also ist \(v\) eine Einheit in \(C\),
			also ist \(\chi(v) \neq 0\). Definiere \(\psi\coloneqq \chi|_B\colon B \to L\).
			\qedhere
		\end{enumerate}
	\end{proof}
\end{frame}

\begin{frame}{Schwache Form des Hilbertschen Nullstellensatzes}
	Sei \(K\) ein Körper.
	\begin{corollary}<+->
		\label{cor:weak_hilbert1}
		Ist eine endlich erzeugte \(K\)-Algebra \(B\) ein Körper, so ist \(B\)
		eine endliche algebraische Erweiterung von \(K\).
	\end{corollary}
	\begin{proof}<+->
		Sei \(L\) ein algebraischer Abschluß von \(K\). Nach der Proposition
		läßt sich die Körpererweiterung \(K \subset L\) zu einer
		Körpererweiterung \(B \subset L\) fortsetzen. Damit ist \(B\)
		algebraisch über \(K\) und als endlich erzeugte Algebra damit auch
		endlich.
	\end{proof}
	\begin{corollary}[Schwacher Hilbertscher Nullstellensatz]<+->
		Ist \(\ideal m\) ein maximales Ideal einer endlich erzeugten
		\(K\)-Algebra \(A\), so ist \(A/\ideal m\)
		eine endliche algebraische Erweiterung von \(K\). Insbesondere
		ist \(A/\ideal m \cong K\), falls \(K\) algebraisch abgeschlossen ist.
		\qed
	\end{corollary}
\end{frame}

