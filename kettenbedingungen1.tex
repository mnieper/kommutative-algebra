\section{Kettenbedingungen I}

\subsection{Kettenbedingungen}

\begin{frame}{Stationäre Folgen}
	\begin{proposition}<+->
		Sei \(X\) eine teilweise geordnete Menge. Dann sind folgende beide
		Bedingungen an \(X\) äquivalent:
		\begin{enumerate}[<+->]
		\item<.->
			Jede aufsteigende Folge \(x_1 \le x_2 \le \dotsb\) in \(X\) 
			ist stationär, das heißt es existiert ein \(n\) mit
			\(x_n = x_{n + 1} = \dotsb\).
		\item
			Jede nicht leere Teilmenge von \(X\) besitzt ein maximales Element.
		\end{enumerate}
	\end{proposition}
	\begin{proof}<+->
		\begin{enumerate}[<+->]
		\item<.->
			Angenommen, es gibt eine nicht leere Teilmenge ohne maximales
			Element. Dann können wir induktiv eine unendliche strikt
			aufsteigende Folge von Elementen in \(X\) konstruieren.
		\item
			Besitzt jede nicht leere Teilmenge ein maximales Element, so
			auch die Menge \((x_m)_{m \in \set N}\).
			\qedhere
		\end{enumerate}
	\end{proof}
\end{frame}

\begin{frame}{Kettenbedingungen für Untermoduln}
	Sei \(A\) ein Ring. Sei \(M\) ein \(A\)-Modul.
	\begin{definition}<+->
		Ist jede bezüglich der Inklusion aufsteigende Folge
		\(N_1 \subset N_2 \subset \dotsb\)
		von Untermoduln von \(M\) stationär, heißt \(M\) \emph{noethersch}.
	\end{definition}
	\begin{definition}<+->
		Ist jede bezüglich der Inklusion absteigende Folge
		\(N_1 \supset N_2 \supset \dotsb\) von Untermoduln von \(M\)
		stationär, heißt \(N\) \emph{artinsch}.
	\end{definition}
\end{frame}

\begin{frame}{Beispiele für noethersche und artinsche Moduln}
	\begin{example}<+->
		Jede endliche abelsche Gruppe ist als \(\set Z\)-Modul sowohl
		noethersch als auch artinsch.
	\end{example}
	\begin{example}<+->
		Der Ring \(\set Z\) der ganzen Zahlen ist als \(\set Z\)-Modul
		noethersch, aber nicht artinsch, denn wir haben
		\((a) \supsetneq (a^2) \supsetneq \dotsb\) für \(a \in \set Z
		\setminus \{0, \pm 1\}\).
	\end{example}
\end{frame}

\begin{frame}{Weitere Beispiele für noethersche und artinsche Moduln}
	Sei \(p\) eine Primzahl. Sei \(G\) die Gruppe der Elemente von
	\(\set Q/\set Z\) mit \(p\)-Potenzordnung. 
	\begin{example}<+->
		Es besitzt \(G\) genau
		eine Untergruppe \(G_n\) der Ordnung \(p^n\) für alle \(n \ge 0\)
		und \(G_0 \subsetneq G_1 \subsetneq \dotsb\), so daß \(G\) nicht 
		noethersch ist. Auf der anderen Seite sind die einzigen echten
		Untergruppen von \(G\) die \(G_n\), so daß \(G\) artinsch ist.
	\end{example}
	\begin{example}<+->
		Sei \(H\) die Gruppe aller rationalen Zahlen, in deren gekürzter
		Bruchdarstellung der Nenner eine \(p\)-Potenz ist. Dann ist \(H\)
		weder noethersch noch artinsch, denn es gibt eine exakte Sequenz
		\(0 \to \set Z \to H \to G \to 0\) und \(\set Z\) ist nicht artinsch
		und \(G\) ist nicht noethersch.
	\end{example}
\end{frame}

\begin{frame}{Beispiele mit Polynomringen}
	Sei \(K\) ein Körper.
	\begin{example}<+->
		Der Polynomring \(K[x]\) ist als Modul über sich
		selbst noethersch, das heißt, er erfüllt die Kettenbedingung für seine
		Ideale. Auf der anderen Seite ist er nicht artinsch.
	\end{example}
	\begin{example}<+->
		Der Polynomring \(K[x_1, x_2, \dotsc]\) in unendlich vielen Variablen
		ist weder noethersch noch artinsch: Die Folgen
		\((x_1) \subsetneq (x_1, x_2) \subsetneq \dotsb\) ist streng aufsteigend,
		die Folge \((x_1) \supsetneq (x_1^2) \supsetneq \dotsb\) ist streng
		absteigend.
	\end{example}
\end{frame}

\begin{frame}{Untermoduln noetherscher Moduln sind endlich erzeugt}
	\begin{proposition}<+->
		Sei \(A\) ein Ring. Ein \(A\)-Modul \(M\) ist genau dann noethersch,
		wenn alle seine Untermoduln endlich erzeugt sind.
	\end{proposition}
	\begin{proof}<+->
		\begin{enumerate}[<+->]
		\item<.->
			Sei \(N\) ein Untermodul eines noetherschen \(A\)-Moduls. Dann gibt es
			einen maximalen endlich erzeugten Untermodul \(N_0\) von \(N\). Aus der
			Maximalität folgt \(N_0 = N_0 + Ax\) für alle \(x \in N\), also \(N_0 = N\).
		\item
			Sei umgekehrt  jeder Untermodul von \(M\) endlich erzeugt, und sei
			\(M_1 \subset M_2 \subset \dotsb\) eine aufsteigende Kette von Untermoduln von \(M\).
			Dann ist \(N \coloneqq \bigcup\limits_{n = 1}^\infty M_i\) ein endlich erzeugter
			Untermodul. Alle Erzeuger sind schon in einem \(M_n\) enthalten, also \(M_n = N\),
			und damit ist die Kette stationär.
			\qedhere
		\end{enumerate}
	\end{proof}
\end{frame}

\mode<article>{Diese Proposition ist der Grund dafür, warum noethersche Moduln so wichtig sind.
Es stellt sich heraus, daß dies in der Praxis häufig die richtige Endlichkeitsbedingung ist.}

\begin{frame}{Noethersche und artinsche Moduln in exakten Sequenzen}
	\begin{proposition}<+->
		Sei \(A\) ein Ring. Sei \(0 \to M' \to M \to M'' \to 0\) eine exakte Sequenz
		von \(A\)-Moduln. Dann ist \(M\) genau dann noethersch, wenn \(M'\) und \(M''\)
		noethersch sind.
	\end{proposition}
	\begin{proof}<+->
		\begin{enumerate}[<+->]
		\item<.->
			Jede Kette von Untermoduln in \(M'\) bzw.\ \(M''\) liefert eine Kette
			von Untermoduln in \(M\). Ist sie dort stationär, so ist sie auch in
			\(M'\) bzw.\ \(M''\) stationär.
		\item
			Ist \((M_n)\) eine Kette von Untermoduln in \(M\) und ist ihr Urbild in \(M'\) und
			ihr Bild in \(M''\) stationär, so ist sie selbst stationär.
			\qedhere
		\end{enumerate}
	\end{proof}
	\begin{remark}<+->
		Eine entsprechende Aussage gilt auch für artinsche anstelle noetherscher Moduln.
	\end{remark}
\end{frame}

\begin{frame}{Endliche Summen noetherscher und artinscher Moduln}
	\begin{corollary}<+->
		Sei \(A\) ein Ring. Sind dann \(M_1, \dotsc, M_n\) noethersche \(A\)-Moduln,
		so ist auch \(\bigoplus\limits_{i = 1}^n M_i\) ein noetherscher \(A\)-Modul.
	\end{corollary}
	\begin{proof}<+->
		Wende Induktion und die Proposition auf die exakte Sequenz
		\[
			0 \to M_1 \to \bigoplus\limits_{i = 1}^n M_i \to \bigoplus_{i = 2}^n M_i \to 0
		\]
		an.
	\end{proof}
	\begin{remark}<+->
		Eine entsprechende Aussage gilt auch für artinsche anstelle noetherscher Moduln.
	\end{remark}
\end{frame}

\subsection{Noethersche und artinsche Ringe}

\begin{frame}{Definition noetherscher und artinscher Ringe}
	\begin{definition}<+->
		Ein Ring \(A\) heißt \emph{noethersch}, falls \(A\) als Modul über sich selbst
		noethersch ist.
	\end{definition}
	\begin{definition}<+->
		Ein Ring \(A\) heißt \emph{artinsch}, falls \(A\) als Modul über sich selbst
		artinsch ist.
	\end{definition}
	\begin{remark}<+->
		Eine Ring \(A\) ist also genau dann noethersch bzw.~artinsch, wenn jede Kette aufsteigender
		bzw.~absteigender Ideale in \(A\) stationär ist.
	\end{remark}
\end{frame}

\begin{frame}{Beispiele zu noetherschen und artinschen Ringen}
	\begin{example}<+->
		Jeder Körper ist sowohl ein noetherscher und artinscher Ring. Dasselbe gilt für die
		Ringe \(\set Z/(n)\) mit \(n > 0\).
	\end{example}
	\begin{example}<+->
		Der Ring \(\set Z\) der ganzen Zahlen ist noethersch, aber nicht artinsch.
	\end{example}
	\begin{example}<+->
		Jeder Hauptidealbereich ist noethersch, denn alle seine Ideale sind endlich erzeugt.
	\end{example}
\end{frame}

\begin{frame}{Beispiele nicht noetherscher Ringe}
	\begin{example}<+->
		Sei \(K\) ein Körper. Der Polynomring \(K[x_1, x_2, \dotsc]\) in unendlich vielen
		Variablen ist nicht noethersch. Als Integritätsbereich ist er aber in einem Körper
		enthalten. Damit sehen wir, daß ein Unterring eines noetherschen Ringes im allgemeinen
		nicht wieder noethersch sein muß.
	\end{example}
	\begin{example}<+->
		Sei \(X\) ein kompakter Hausdorffraum, und sei \(F_1 \supsetneq F_2 \supsetneq \dotsb\)
		eine streng absteigende Folge abgeschlossener Teilmengen. Sei \(\Cont(X)\) der Ring
		der stetigen reellwertigen Funktionen auf \(X\). Dann definieren die
		\(\ideal a_n \coloneqq \{f \in \Cont(X) \mid f|_{F_n} = 0\}\) eine streng aufsteigende
		Folge von Idealen in \(\Cont(X)\), es ist \(\Cont(X)\) also nicht noethersch.
	\end{example}
\end{frame}

\begin{frame}{Endlich erzeugte Moduln noetherscher und artinscher Ringe}
	\begin{proposition}<+->
		Sei \(A\) ein noetherscher Ring. Ist dann \(M\) ein endlich erzeugter
		\(A\)-Modul, so ist \(M\) noethersch.
	\end{proposition}
	\begin{proof}<+->
		Es ist \(M\) ein Quotient des \(A\)-Moduls \(A^n\), und Quotienten und Summen
		noetherscher Moduln sind wieder noethersch.
	\end{proof}
	\begin{remark}<+->
		Eine entsprechende Aussage gilt auch für artinsche anstelle von noetherschen
		Ringen.
	\end{remark}
\end{frame}

\begin{frame}{Quotienten noetherscher und artinscher Ringe}
	\begin{proposition}<+->
		Sei \(\ideal a\) ein Ideal eines noetherschen Ringes \(A\). Dann
		ist auch \(A/\ideal a\) ein noetherscher Ring.
	\end{proposition}
	\begin{proof}<+->
		Da Quotienten noetherscher Moduln noethersch sind, ist \(A/\ideal a\)
		ein noetherscher \(A\)-Modul. Damit ist \(A/\ideal a\) aber auch
		noethersch über sich selbst.
	\end{proof}
	\begin{remark}<+->
		Eine entsprechende Aussage gilt auch für artinsche anstelle von noetherschen
		Ringen.
	\end{remark}
\end{frame}

