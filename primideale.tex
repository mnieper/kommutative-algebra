\section{Primideale und maximale Ideale}

\subsection{Primideale}

\begin{frame}{Definition von Prim- und maximalen Idealen}
    Seien \(\ideal p, \ideal m\) zwei Ideale eines kommutativen Ringes \(A\).
    \begin{definition}<+->
        \begin{enumerate}[<+->]
        \item<.->
            Das Ideal \(\ideal p\) von \(A\) heißt \emph{prim},
            falls \(1 \notin \ideal p\) und falls
            aus \(a b \in \ideal p\) für \(a, b \in A\) schon \(a \in \ideal p\)
            oder \(b \in \ideal p\) folgt.
        \item
            Das Ideal \(\ideal m\) von \(A\) heißt \emph{maximal},
            falls für jedes Ideal \(\ideal a\) von \(A\)
            mit \(\ideal a \supset \ideal m\) \alert{entweder} \(\ideal a = \ideal m\)
            \alert{oder} \(\ideal a = (1)\) gilt.
        \end{enumerate}
    \end{definition}
    \begin{proposition}<+->
        Das Ideal \(\ideal p\) ist genau dann prim, wenn \(A/\ideal p\) ein 
        Integritätsbereich ist.
        \qed
    \end{proposition}
    \begin{example}<+->
        Das Nullideal ist genau dann prim, wenn \(A\) ein Integritätsbereich ist.
    \end{example}
\end{frame}

\begin{frame}{Beispiele von Primidealen}
    \begin{example}<+->
        Jedes Ideal in \(\set Z\) ist von der Form \((m)\) mit \(m \ge 0\). Das 
        Ideal \((m)\) ist genau dann prim, falls \(m = 0\) oder \(m\) eine Primzahl
        ist. Im Falle, daß \(p\) eine Primzahl ist, ist \((p)\) ein maximales Ideal.
    \end{example}
    \begin{example}<+->
        Sei \(f \in A \coloneqq K[x_1, \dotsc, x_n]\) ein irreduzibles Polynom über
        dem Körper \(K\). Da \(A\) ein faktorieller Ring ist, ist \((f)\) ein 
        Primideal von \(A\).
        
        Das Ideal \((x_1, \dotsc, x_n)\) ist ein maximales Ideal in \(A\).
    \end{example}
    \begin{proposition}<+->
        Sei \(\ideal a\) ein Ideal eines kommutativen Ringes \(A\) und
        \(\pi\colon A \to A/\ideal a\) der kanonische Homomorphismus.
        Durch \(\ideal p = \pi^{-1}(\bar{\ideal p})\) wird eine bijektive 
        ordnungserhaltende Korrespondenz zwischen den Primidealen \(\ideal p\)
        von \(A\) mit \(\ideal p \supset \ideal a\) und den Primidealen
        \(\bar{\ideal p}\) von \(A/\ideal a\) gegeben.
        \qed
    \end{proposition}
\end{frame}

\begin{frame}{Primideale in Hauptidealbereichen}
    \begin{proposition}<+->
        Sei \(\ideal p\) ein Primideal in einem Hauptidealbereich \(A\). Ist
        \(\ideal p \neq (0)\), so ist \(\ideal p\) ein maximales Ideal.
    \end{proposition}
    \begin{proof}<+->
        \begin{enumerate}[<+->]
        \item<.->
            Sei \(\ideal p = (p)\) mit \(p \neq 0\). Sei \((q) \supset (p)\). Wir
            müssen zeigen, daß \(q\) eine Einheit ist oder daß \((q) = (p)\).
        \item
            Da \((q) \supset (p)\), existiert ein \(x \in A\) mit \(p = x q\).
        \item
            Aus \(x q \in (p)\) folgt \(x \in (p)\) oder \(q \in (p)\).
        \item
            Sei \(x \in (p)\). Dann existiert ein \(y \in A\) mit \(x = y p\), also
            \(p = y q p\). Da \(p \neq 0\), folgt \(y q = 1\). Also ist \(q\) eine
            Einheit.
        \item
            Sei \(q \in (p)\). Dann ist \((q) \subset (p)\) und damit \((q) = (p)\).
            \qedhere
        \end{enumerate}
    \end{proof}
\end{frame}

\subsection{Maximale Ideale}

\begin{frame}{Maximale Ideale}
    Sei \(A\) ein kommutativer Ring.
    \begin{proposition}<+->
        Sei \(\ideal a\) ein Ideal von \(A\) und \(\pi\colon A \to A/\ideal a\) der
        kanonische Homomorphismus.
        Durch \(\ideal m = \pi^{-1}(\bar{\ideal m})\) wird eine bijektive 
        ordnungserhaltende Korrespondenz zwischen den maximalen Idealen \(\ideal m\)
        von \(A\) mit \(\ideal m \supset \ideal a\) und den maximalen Idealen
        \(\bar{\ideal m}\) von \(A/\ideal a\) gegeben.
        \qed
    \end{proposition}
    \begin{proposition}<+->
        Ein Ideal \(\ideal m\) von \(A\) ist genau dann maximal, wenn
        \(A/\ideal m\) ein Körper ist.
        \qed
    \end{proposition}
    \begin{corollary}<+->
        Jedes maximale Ideal von \(A\) ist prim.
        \qed
    \end{corollary}
\end{frame}

\begin{frame}{Existenz maximaler Ideale}
    \begin{theorem}<+->
        Ein Ring \(A\) besitzt genau dann ein maximales Ideal, wenn \(A\) nicht der
        Nullring ist.
    \end{theorem}
    \begin{proof}<+->
        \begin{enumerate}[<+->]
        \item<.->
            Ein maximales Ideal von \(A\) ist gerade ein maximales Element des
            Systems \(\mathfrak A\) aller Ideale \(\ideal a\) von \(A\) mit
            \(\ideal a \neq (1)\) bezüglich der Inklusionsordnung.
        \item
            Jede Kette \(\mathfrak C\) in \(\mathfrak A\) besitzt eine obere Schranke
            in \(\mathfrak A\), nämlich \(\bigcup \mathfrak C\).
        \item
            Nach dem Zornschen Lemma besitzt \(\mathfrak A\) damit genau dann ein
            maximales Element, falls \(\mathfrak A\) nicht leer ist.
        \item
            Das System \(\mathfrak A\) ist genau nicht leer, wenn \(A\) nicht der
            Nullring ist, denn dann ist \((0) \in \mathfrak A\).
            \qedhere
        \end{enumerate}
    \end{proof}
\end{frame}

\begin{frame}{Folgerungen aus der Existenz maximaler Ideale}
    Sei \(A\) ein kommutativer Ring.
    \begin{corollary}<+->
        Sei \(\ideal a\) ein Ideal von \(A\). Dann existiert genau dann ein
        maximales Ideal \(\ideal m\) von \(A\) mit \(\ideal a \subset \ideal m\),
        wenn \(\ideal a \neq (1)\).
    \end{corollary}
    \begin{proof}<+->
        Der Ring \(A/\ideal a\) besitzt genau dann ein maximales Ideal, wenn
        \(\ideal a \neq (1)\). Ein solches maximales Ideal entspricht einem maximalen
        Ideal \(\ideal m\) von \(A\) mit \(\ideal m \supset \ideal a\).
    \end{proof}
    \begin{corollary}<+->
        Ein Element \(x\) von \(A\) liegt genau dann in einem maximalen Ideal von
        \(A\), wenn \(x\) keine Einheit ist.
        \qed
    \end{corollary}
\end{frame}

\subsection{Lokale Ringe}

\begin{frame}{Lokale Ringe}
    \begin{definition}<+->
        Ein \emph{lokaler Ring \(A\)} ist ein kommutativer Ring mit genau einem
        maximalen Ideal \(\ideal m\). Der Körper \(A/\ideal m\) ist der
        \emph{Restklassenkörper von \(A\)}.
    \end{definition}
    \begin{example}<+->
        Körper sind lokale Ringe.
    \end{example}
    \begin{notation}<+->
        Ist \(A\) ein lokaler Ring mit maximalem Ideal \(\ideal m\) und
        Restklassenkörper \(F\), so schreiben wir häufig \((A, \ideal m)\) oder
        \((A, \ideal m, F)\) für \(A\).
    \end{notation}
    \begin{visibleenv}<+->
        Die folgende Definition verallgemeinert den Begriff des lokalen Ringes.
    \end{visibleenv}
    \begin{definition}<.->
        Ein \emph{halblokaler Ring \(A\)} ist ein kommutativer Ring mit nur
        endlich vielen maximalen Idealen.
    \end{definition}
\end{frame}

\begin{frame}{Ringe, in denen die Nichteinheiten ein Ideal bilden}
    \begin{proposition}<+->
        Sei \(\ideal m\) ein Ideal eines kommutativen Ringes \(A\), so daß
        \(A \setminus \ideal m = A^\units\). Dann ist \((A, \ideal m)\) ein
        lokaler Ring.
    \end{proposition}
    \begin{proof}<+->
        Sei \(\ideal a \neq (1)\) ein Ideal von \(A\). Wegen \(1 \neq \ideal a\)
        enthält \(\ideal a\) keine Einheiten. Also ist
        \(\ideal a \subset A \setminus A^\units = \ideal m\). Folglich
        ist \(\ideal m\) ein maximales Ideal und das einzige.
    \end{proof}
\end{frame}

\begin{frame}{Ein Kriterium für einen lokalen Ring}
    \begin{proposition}<+->
        Sei \(\ideal m\) ein maximales Ideal eines kommutativen Ringes \(A\),
        so daß jedes Element von \(1 + \ideal m\) eine Einheit in \(A\) ist.
        Dann ist \(A\) ein lokaler Ring.
    \end{proposition}
    \begin{proof}<+->
        \begin{enumerate}[<+->]
        \item<.->
            Wir zeigen, daß \(A \setminus \ideal m = A^\units\). Sei dazu
            \(x \in A \setminus \ideal m\).
        \item
            Da \(\ideal m\) ein maximales Ideal ist, ist das von \(\ideal m\)
            und \(x\) aufgespannte Ideal das Einsideal.
        \item
            Damit existieren ein \(y \in A\) und ein \(t \in \ideal m\) mit
            \(x y + t = 1\), also \(x y = 1 - t \in 1 + \ideal m\).
        \item
            Nach Voraussetzung ist \(x y\) damit eine Einheit. Folglich ist
            auch \(x \in A^\units\).
            \qedhere
        \end{enumerate}
    \end{proof}
\end{frame}

\subsection{Bezeichnungen}

\begin{frame}{Bezeichnungen für Primideale}
    \begin{convention}<+->
        Primideale bezeichnen wir mit kleinen deutschen Buchstaben
        \(\ideal p, \ideal q, \dotsc\).
    \end{convention}
    \begin{convention}<+->
        Maximale Ideale bezeichnen wir mit kleinen deutschen Buchstaben
        \(\ideal m, \ideal n, \dotsc\).
    \end{convention}    
\end{frame}

