\section{Noethersche Ringe}

\subsection{Elementare Eigenschaften noetherscher Ringe}

\begin{frame}{Bilder noetherscher Ringe}
	Wir erinnern an folgende Möglichkeiten, einen Ring \(A\) als noethersch zu
	charakterisieren:
	\begin{enumerate}[<+->]
	\item
		Jede nicht leere Menge von Idealen von \(A\) besitzt ein maximales Element.
	\item
		Jede aufsteigende Kette von Idealen in \(A\) ist stationär.
	\item
		Jedes Ideal in \(A\) ist endlich erzeugt.
	\end{enumerate}
	\begin{proposition}<+->
		Sei \(\phi\colon A \to B\) ein Homomorphismus von Ringen. Ist \(A\)
		noethersch, so ist auch \(\phi(A)\) noethersch.
	\end{proposition}
	\begin{proof}<+->
		Nach dem Homomorphiesatz ist \(\phi(A) \cong A/\ideal \ker \phi\), und
		Quotienten noetherscher Ringe sind noethersch.
	\end{proof}
\end{frame}

\begin{frame}{Endliche Erweiterungen noetherscher Ringe}
	\begin{proposition}<+->
		Sei \(A \subset B\) eine endliche Erweiterung kommutativer Ringe.
		Sei \(A\) noethersch. Dann ist auch \(B\) noethersch.
	\end{proposition}
	\begin{proof}<+->
		Da \(B\) als \(A\)-Modul endlich erzeugt ist und \(A\) ein noetherscher Ring
		ist, ist \(B\) als \(A\)-Modul noethersch. Damit ist \(B\) auch als
		Modul über sich selbst noethersch.
	\end{proof}
	\begin{example}<+->
		Der Ring \(\set Z[i]\) der ganzen Gaußschen Zahlen eine endliche Erweiterung
		von \(\set Z\) und damit noethersch.
	\end{example}
	\begin{remark}<+->
		Allgemeiner ist ganze Abschluß von \(\set Z\) in einer beliebigen endlichen
		algebraischen Erweiterung von \(\set Q\) ein noetherscher Ring.
	\end{remark}
\end{frame}

\begin{frame}{Lokalisierungen noetherscher Ringe}
	Sei \(A\) ein noetherscher kommutativer Ring. 
	\begin{proposition}<+->
		Sei \(S \subset A\) multiplikativ
		abgeschlossen. Dann ist \(S^{-1} A\) noethersch.
	\end{proposition}
	\begin{proof}<+->
		\begin{enumerate}[<+->]
		\item<.->
			Jedes Ideal in \(S^{-1} A\) ist von der Form \(S^{-1} \ideal a\), wobei
			\(\ideal a\) ein Ideal in \(A\) ist.
		\item
			Erzeugen \(x_1, \dotsc, x_n\) das Ideal \(\ideal a\), so wird \(S^{-1} \ideal a\)
			von \(\frac {x_1} 1, \dotsc, \frac{x_n} 1\) erzeugt.
			\qedhere
		\end{enumerate}
	\end{proof}
	\begin{corollary}<+->
		Die Halme \(A_{\ideal p}\) von \(A\) sind noethersch.
		\qed
	\end{corollary}
\end{frame}

\subsection{Der Hilbertsche Basissatz}

\begin{frame}{Der Hilbertsche Basissatz}
	\begin{theorem}[Hilbertscher Basissatz]<+->
		Ist \(A\) ein noetherscher kommutativer Ring, so ist auch der Polynomring \(A[x]\) noethersch.
	\end{theorem}
\end{frame}

\begin{frame}{Beweis des Hilbertschen Basissatzes}
	\begin{proof}<+->
		\begin{enumerate}[<+->]
		\item<.->
			Sei \(\ideal a\) ein Ideal in \(A[x]\). Die führenden Koeffizienten der
			Polynome in \(\ideal a\) erzeugen ein Ideal \(\ideal l\) in \(A\), welches
			nach Voraussetzung von endlich vielen Elementen \(a_1, \dotsc, a_n\) erzeugt ist.
		\item
			Zu jedem \(a_i\) wählen wir ein Polynom \(f_i \in \ideal a\) mit Leitmonom
			\(a_i x^{r_i}\). Sei \(r\) das Maximum der \(r_i\). Die \(f_i\)
			erzeugen ein Ideal \(\ideal a' \subset \ideal a\).
		\item
			Sei \(f \in \ideal a\) ein Polynom mit Leitmonom \(a x^m\).
			Ist \(m \ge r\), so existieren \(u_i \in A\) mit \(a = \sum\limits_i u_i a_i\).
			Dann ist \(f - \sum\limits_i u_i f_i x^{m - r_i} \in \ideal a\) ein Polynom echt kleineren Grades 
			als \(f\). Indem wir so fortfahren, können wir \(f = g + h\) mit \(h \in \ideal a'\)
			schreiben, wobei \(\deg g < r\).
		\item
			Ist \(M\) der von \(1, x, \dots, x^{r - 1}\) erzeugte \(A\)-Modul, so haben
			wir \(\ideal a = (\ideal a \cap M) + \ideal a'\) gezeigt.
		\item
			Es ist \(\ideal a \cap M\) als Untermodul eines noetherschen Moduls endlich erzeugt, 
			etwa von \(g_1, \dotsc, g_m\). Damit ist \(\ideal a\) von
			\(g_1, \dotsc, g_m, f_1, \dotsc, f_n\) erzeugt.
			\qedhere
		\end{enumerate}
	\end{proof}
\end{frame}

\begin{frame}{Folgerungen aus dem Hilbertschen Basissatz}
	\begin{corollary}<+->
		Ist \(A\) ein noetherscher kommutativer Ring, so ist auch der Polynomring \(A[x_1, \dotsc, x_n]\) in
		endlich vielen Variablen noethersch.
	\end{corollary}
	\begin{proof}<+->
		Folgt aus dem Hilbertschen Basissatz mittels Induktion über \(n\).
	\end{proof}
	\begin{corollary}<+->
		Sei \(A\) ein noetherscher Ring. Dann ist jede endlich erzeugte kommutative
		\(A\)-Algebra \(B\) noethersch.
	\end{corollary}
	\begin{proof}<+->
		Es ist \(B\) ein Bild eines Polynomringes der Form \(A[x_1, \dotsc, x_n]\) unter einem
		Ringhomomorphismus.
	\end{proof}
\end{frame}

\begin{frame}{Zwischenringe in endlich erzeugten Erweiterungen}
	\begin{proposition}<+->
		\label{prop:intermediate_ring}
		Seien \(A \subset B \subset C\) Erweiterungen kommutativer Ringe. Seien \(A\) noethersch
		und \(C\) endlich erzeugt als \(A\)-Algebra. Sei weiter \(C\) endlich als \(B\)-Algebra.
		Dann ist \(B\) endlich erzeugt als \(A\)-Algebra.
	\end{proposition}
	\begin{remark}<+->
		In der Situation der Proposition ist es wegen~\prettyref{prop:characterization-integral-elements} äquivalent zu fordern, daß \(C\) ganz als \(B\)-Algebra ist.
	\end{remark}
\end{frame}

\begin{frame}{Beweis der Proposition über Zwischenringe}
	\begin{proof}<+->
		\begin{enumerate}[<+->]
		\item<.->
			Seien \(x_1, \dotsc, x_m\) Erzeuger von \(C\) als \(A\)-Algebra. Seien weiter
			\(y_1, \dotsc, y_n\) Erzeuger von \(C\) als \(B\)-Modul. Damit existieren
			\(b_i^j \in B\) mit \(x_i = \sum\limits_j b_i^j y_j\) und \(b_{ij}^k \in B\)
			mit \(y_i y_j = \sum\limits_k b_{ij}^k y_k\). Sei \(B_0\) die von den \(b_i^j\) und
			\(b_{ij}^k\) erzeugte \(A\)-Unteralgebra von \(B\). Da \(A\)
			noethersch ist, ist auch \(B_0\) noethersch.
		\item
			Jedes Element in \(C\) ist ein Polynom in den \(x_i\) mit Koeffizienten in \(A\).
			Durch wiederholtes Einsetzen der Ausdrücke für \(x_i\) und \(y_i y_j\) erhalten wir,
			daß jedes Element in \(C\) eine Linearkombination der \(y_j\) mit Koeffizienten
			in \(B_0\) ist, und damit ist \(C\) eine endliche \(B_0\)-Algebra.
		\item
			Da \(B_0\) noethersch ist, ist damit \(C\) ein noetherscher \(B_0\)-Modul.
			Da \(B\) ein Untermodul von \(C\) ist, ist \(B\) als \(B_0\)-Modul endlich erzeugt.
		\item
			Schließlich ist \(B_0\) als \(A\)-Algebra endlich erzeugt, also ist
			auch \(B\) als \(A\)-Algebra endlich erzeugt.
		\qedhere
		\end{enumerate}
	\end{proof}
\end{frame}

