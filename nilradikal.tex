\section{Das Nil- und das Jacobsonsche Radikal}

\subsection{Das Nilradikal}

\begin{frame}{Definition des Nilradikals}
    \begin{proposition}<+->
        Sei \(A\) ein kommutativer Ring. Die Menge \(\ideal n\) der nilpotenten
        Elemente von \(A\) ist ein Ideal, das \alert{Nilradikal von \(A\)}. Der Ring
        \(A/\ideal n\) hat außer \(0\) keine nilpotenten Elemente.
    \end{proposition}
    \begin{proof}<+->
        \begin{enumerate}[<+->]
        \item<.->
            Es ist \(0 \in \ideal n\). Ist \(x \in \ideal n\) und \(a \in A\), so
            ist auch \(a x \in \ideal n\), denn \((a x)^n = a^n x^n = 0\) für
            \(n \gg 0\).
        \item   
            Seien \(x, y \in \ideal n\), etwa \(x^m = 0\) und \(y^n = 0\) für
            \(n, m \ge 0\). Nach dem in jedem kommutativen Ring gültigen Binomialsatz
            ist dann
            \((x + y)^{m + n - 1} = \sum\limits_{r + s = m + n - 1} \binom{m + n - 1} r
            x^r y^s = 0\). Also folgt \(x + y \in \ideal n\).
        \item
            Sei \(x \in A\). Sei \(x\) nilpotent in \(A/\ideal n\), das heißt
            \(x^n \in \ideal n\) für \(n \gg 0\). Nach Definition von \(\ideal n\)
            ist \(x^{kn} = (x^n)^k = 0\) für \(k \gg 0\). Folglich ist auch
            \(x \in \ideal n\). Damit ist \(x = 0\) in \(A/\ideal n\).
            \qedhere
        \end{enumerate}
    \end{proof}
\end{frame}

\begin{frame}{Der Schnitt aller Primideale}
    \begin{proposition}<+->
        Das Nilradikal \(\ideal n\) eines kommutativen Ringes \(A\) ist der Schnitt
        \(\bigcap \ideal p\) aller seiner Primideale.
    \end{proposition}
    \begin{proof}<+->
        \begin{enumerate}[<+->]
        \item<.->
            Sei \(f \in \ideal n\), also \(f^n = 0\) für \(n \gg 0\). Ist \(\ideal p\)
            ein Primideal, so folgt aus \(f^n = 0 \in \ideal p\) schon \(f \in \ideal p\),
            also \(f \in \bigcap \ideal p\).
        \item
            Sei umgekehrt \(f \notin \ideal n\). Sei \(\mathfrak A\) die Menge der
            Ideale \(\ideal a\) von \(A\) mit \(f^n \notin \ideal a\) für alle
            \(n \ge 0\). Da \((0) \in \mathfrak A\), besitzt \(\mathfrak A\) nach dem 
            Zornschen Lemma ein maximales Element \(\ideal p \neq (1)\). Insbesondere
            \(f \notin \ideal p\).
        \item
            Es reicht dann zu zeigen, daß \(\ideal p\) prim ist. Dazu seien \(x, y
            \in A \setminus \ideal p\) gegeben. Nach Maximalität von \(\ideal p\)
            ist \(f^m\) im von \(\ideal p\) und \(x\) aufgespannten Ideal
            \(\ideal p + (x)\) für ein
            \(m \ge 0\). Analog ist \(f^n \in \ideal p + (y)\) für ein \(n \ge 0\).
            Es folgt \(f^{m + n} \in \ideal p + (xy)\). Damit ist \(xy \notin \ideal p\).
            \qedhere
        \end{enumerate}
    \end{proof}
\end{frame}

\subsection{Das Jacobsonsche Radikal}

\begin{frame}{Das Jacobsonsche Radikal}
    Sei \(A\) ein kommutativer Ring.
    \begin{definition}<+->
        Das \emph{Jacobsonsche Radikal von \(A\)} ist der Schnitt aller maximalen 
        Ideale von \(A\).
    \end{definition}
    \begin{proposition}
        Ein Element \(x\) von \(A\) ist genau dann im Jacobsonschen Radikal
        \(\ideal j\), wenn \(1 - xy\) für alle \(y \in A\) eine Einheit ist.
    \end{proposition}
    \begin{proof}<+->
        \begin{enumerate}[<+->]
        \item<.->
            Sei \(1 - xy\) keine Einheit. Dann ist \(1 - xy \in \ideal m\) für ein
            maximales Ideal \(\ideal m\). Da \(1 \notin \ideal m\), folgt
            \(xy \notin \ideal m\), also \(x \notin \ideal j\).
        \item
            Sei umgekehrt \(x \notin \ideal j\), also \(x \notin \ideal m\) für ein
            maximales Ideal \(\ideal m\). Damit erzeugen \(x\) und \(\ideal m\) das
            Einsideal, also existiert ein \(y \in A\), so daß \(1 - xy \in \ideal m\).
            Damit ist \(1 - xy\) keine Einheit.
            \qedhere
        \end{enumerate}
    \end{proof}
\end{frame}

\begin{frame}{Interpretation des Jacobsonschen Radikals}
    \begin{visibleenv}<+->
        Das Jacobsonsche Radikal besteht also aus allen Elementen eines kommutativen
        Ringes, welche in einem gewissen Sinne "`nahe bei \(0\) sind"'.
    \end{visibleenv}
\end{frame}

