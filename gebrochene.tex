\section{Gebrochene Ideale}

\subsection{Gebrochene Ideale}

\begin{frame}{Gebrochene Ideale}
	Sei \(A\) ein Integritätsbereich mit Quotientenkörper \(K\).
	\begin{definition}<+->
		Ein \(A\)-Untermodul \(\ideal r\) von \(K\) heißt \emph{gebrochenes Ideal
		von \(A\)}, falls ein \(x \in A \setminus \{0\}\) mit \(x \ideal r
		\subset A\) existiert.
	\end{definition}
	\begin{example}<+->
		Ein ganzes (d.h.\ gewöhnliches) Ideal in \(A\) ist insbesondere
		ein gebrochenes Ideal.
	\end{example}
	\begin{example}<+->
		Jedes \(u \in K\) definiert ein gebrochenes Ideal \((u) \coloneqq
		A u \subset K\), ein \emph{gebrochenes Hauptideal von \(A\)}.
	\end{example}
\end{frame}

\begin{frame}{Gebrochene Ideale in noetherschen Integritätsbereichen}
	Sei \(A\) ein Integritätsbereich mit Quotientenkörper \(K\).
	\begin{proposition}<+->
		Ist \(\ideal r\) ein endlich erzeugter \(A\)-Untermodul von \(K\),
		so ist \(\ideal r\) ein gebrochenes Ideal.
	\end{proposition}
	\begin{proof}<+->
		Ist \(\ideal r\) von \(x_1, \dotsc, x_n\) erzeugt, so können wir
		\(x_i = \frac{y_i}z\) mit \(y_i \in A\) mit einem \(z \in A\) schreiben.
		Dann ist \(z \ideal r \subset A\).
	\end{proof}
	\begin{proposition}<+->
		Ist \(A\) noethersch, so ist jedes gebrochene Ideal \(\ideal r\)
		als \(A\)-Modul endlich erzeugt.
	\end{proposition}
	\begin{proof}<+->
		Ist \(x \in A \setminus \{0\}\), so daß \(\ideal a \coloneqq x \ideal r\)
		ein ganzes Ideal von \(A\) ist, so ist mit \(\ideal a\) auch
		\(\ideal r = x^{-1} \ideal a\) endlich erzeugt.
	\end{proof}
\end{frame}

\subsection{Invertierbare Ideale}

\begin{frame}{Invertierbare Ideale}
	\begin{definition}<+->
		Sei \(A\) ein Integritätsbereich mit Quotientenkörper \(K\).
		Ein \(A\)-Untermodul \(\ideal r\) von \(K\) heißt \emph{invertierbares
		Ideal}, falls ein \(A\)-Untermodul \(\ideal s\) mit \(\ideal r
		\ideal s = A\) existiert.
	\end{definition}
	\begin{visibleenv}<+->
		Ist \(\ideal r\) ein \(A\)-Untermodul von \(K\), so schreiben wir
		\((1 : \ideal r)\) für den \(A\)-Modul derjenigen \(x \in K\) mit
		\(x \ideal r \subset A\).
		\\
		(Diese Schreibweise möge nicht zur Verwechslung mit dem gewöhnlichen
		Idealquotienten führen.)
	\end{visibleenv}
\end{frame}

\begin{frame}{Inversen invertierbarer Ideale}
	\begin{proposition}<+->
		Sei \(A\) ein Integritätsbereich mit Quotientenkörper \(K\).
		Ist dann \(\ideal r\) ein invertierbares Ideal von \(K\), so existiert
		genau ein \(A\)-Untermodul \(\ideal s\) mit \(\ideal r \ideal s = (1)\),
		nämlich \((1 : \ideal r)\).
	\end{proposition}
	\begin{visibleenv}<.->
		In diesem Falle schreiben wir \(\ideal r^{-1} \coloneqq (1 : \ideal r)\).
	\end{visibleenv}
	\begin{proof}<+->
		Sei \(\ideal s\) mit \(\ideal r\ideal s = (1)\). Dann gilt
		\(\ideal s \subset (1 : \ideal r) = (1 : \ideal r) \ideal r \ideal s
		\subset \ideal s\).
	\end{proof}
\end{frame}

\begin{frame}{Invertierbare Ideale sind gebrochene Ideale}
	\begin{proposition}<+->
		Sei \(A\) ein Integritätsbereich. Jedes invertierbare Ideal
		\(\ideal r\) von \(A\) ist als \(A\)-Modul endlich erzeugt, insbesondere
		also ein gebrochenes Ideal von \(A\).
	\end{proposition}
	\begin{proof}<+->
		\begin{enumerate}[<+->]
		\item<.->
			Aus \(\ideal r \ideal r^{-1} = (1)\) folgt die Existenz endlich vieler
			\(x_i \in \ideal r, y_i \in \ideal r^{-1}\) mit \(\sum\limits_i
			x_i y_i = 1\).
		\item
			Für jedes \(x \in \ideal r\) gilt damit \(x = \sum\limits_i
			(y_i x) x_i\).
		\item
			Wegen \(y_i x \in A\) wird \(\ideal r\) damit als \(A\)-Modul von
			\(x_1, \dotsc, x_n\) erzeugt.
			\qedhere
		\end{enumerate}
	\end{proof}
\end{frame}

\begin{frame}{Die Gruppe der invertierbaren Ideale}
	Sei \(A\) ein Integritätsbereich mit Quotientenkörper \(K\).
	\begin{example}<+->
		Ist \(u \in K^\units\), so ist das Hauptideal \((u)\) ein invertierbares
		Ideal mit \((u)^{-1} = (u^{-1})\).
	\end{example}
	\begin{remark}<+->
		Bezüglich der Idealmultiplikation bilden die invertierbaren Ideale
		eine Gruppe, deren neutrales Element durch \((1)\) gegeben ist.
	\end{remark}
\end{frame}

\begin{frame}{Invertierbarkeit ist eine lokale Eigenschaft}
	\begin{proposition}<+->
		Sei \(A\) ein Integritätsbereich. Für ein gebrochenes Ideal
		\(\ideal r\) von \(A\) sind dann folgende Aussagen äquivalent:
		\begin{enumerate}[<+->]
		\item<.->
			Es ist \(\ideal r\) invertierbar.
		\item
			Es ist \(\ideal r\) endlich erzeugt, und für alle Primideale
			\(\ideal p\) von \(A\) ist \(\ideal r_{\ideal p}\) ein
			invertierbares Ideal von \(A_{\ideal p}\).
		\item
			Es ist \(\ideal r\) endlich erzeugt, und für alle maximalen
			Ideale \(\ideal m\) von \(A\) ist \(\ideal r_{\ideal m}\) ein
			invertierbares Ideal von \(A_{\ideal m}\).
		\end{enumerate}
	\end{proposition}
\end{frame}

\begin{frame}{Beweis, daß Invertierbarkeit eine lokale Eigenschaft ist}
	\begin{proof}<+->
		\begin{enumerate}[<+->]
			\item<.->
				Es ist \(A_{\ideal p} = (\ideal r \cdot (1 : \ideal r))_{\ideal p}
				= \ideal r_{\ideal p} \cdot (1 : \ideal r_{\ideal p})\), da
				\(\ideal r\) als invertierbares Ideal endlich erzeugt ist.
				Damit folgt die zweite aus der ersten Aussage.
			\item
				Die dritte Aussage folgt trivialerweise aus der zweiten Aussage.
			\item
				Um aus der dritten Aussage, die erste zu folgern, betrachten wir
				\(\ideal a = \ideal r \cdot (1 : \ideal r)\), ein ganzes Ideal.
				Für jedes maximale Ideal \(\ideal m\) haben wir
				\(\ideal a_{\ideal m} = \ideal r_{\ideal m}
				\cdot (1 : \ideal r_{\ideal m}) = A_{\ideal m}\), wenn \(\ideal r\) endlich erzeugt
				und \(\ideal r_{\ideal m}\) invertierbar ist. Damit ist
				\(\ideal a = (1)\) und damit \(\ideal r\) invertierbar.
				\qedhere
		\end{enumerate}
	\end{proof}
\end{frame}

\subsection{Gebrochene Ideale in Dedekindschen Bereichen}

\begin{frame}{Gebrochene Ideale in diskreten Bewertungsbereichen}
	\begin{proposition}<+->
		Ein lokaler Integritätsbereich \(A\), welcher kein Körper ist,
		ist genau dann ein diskreter
		Bewertungsring, wenn jedes nicht verschwindende gebrochene Ideal
		von \(A\) invertierbar ist.
	\end{proposition}
	\begin{proof}<+->
		\renewcommand{\qedsymbol}{}
		Sei zunächst \(A\) ein diskreter Bewertungsring. Sei \(x\) ein
		Erzeuger des maximalen Ideals von \(A\). Sei \(\ideal r \neq (0)\)
		ein gebrochenes Ideal. Dann existiert ein \(y \in A\) mit
		\(y \ideal r \subset A\), also ist \(y \ideal r = (x^r)\) für ein
		\(r \in \set N_0\). Damit folgt \(\ideal r = (x^{r - s})\), wenn
		\((y) = (x^s)\). Also ist \(\ideal r\) invertierbar.
	\end{proof}
\end{frame}

\begin{frame}{Fortsetzung des Beweises}
	\begin{proof}[Fortsetzung des Beweises]<+->
		\begin{enumerate}[<+->]
		\item<.->
			Sei umgekehrt jedes nicht verschwindende gebrochene Ideal
			invertierbar. Damit ist insbesondere jedes nicht triviale ganze Ideal
			invertierbar, also auch endlich erzeugt, also ist \(A\) noethersch.
		\item
			Damit reicht es zu zeigen, daß jedes nicht verschwindende Ideal
			eine Potenz des maximalen Ideals \(\ideal m\) ist. Angenommen, dies ist falsch.
			Dann existiert ein maximales nicht verschwindendes Ideal \(\ideal a\),
			welches keine Potenz von \(\ideal m\) ist.
		\item
			Da \(\ideal a \subset \ideal m\) ist \(\ideal m^{-1} \ideal a\) ein
			ganzes Ideal mit \(\ideal m^{-1} \ideal a \supset \ideal a\).
		\item
			Wäre \(\ideal m^{-1} \ideal a = \ideal a\), dann auch \(\ideal a = \ideal m \ideal a\),
			also \(\ideal a = 0\) nach dem Nakayamaschen Lemma.
		\item
			Daher ist \(\ideal m^{-1} \ideal a \supsetneq \ideal a\), also ist
			\(\ideal m^{-1} \ideal a\) aufgrund der Maximalität von \(\ideal a\)
			eine Potenz von \(\ideal m\), also auch \(\ideal a\). Widerspruch.
			\qedhere
		\end{enumerate}
	\end{proof}
\end{frame}

\begin{frame}{Gebrochene Ideale in Dedekindschen Bereichen}
	\begin{theorem}<+->
		Ein Integritätsbereich \(A\), welcher kein Körper ist, ist genau
		dann ein Dedekindscher Bereich, wenn jedes nicht verschwindende
		gebrochene Ideal invertierbar ist.
	\end{theorem}
	\begin{proof}<+->
		\begin{enumerate}[<+->]
		\item<.->
			Sei \(A\) ein Dedekindscher Bereich. Ist \(\ideal r \neq (0)\) ein
			gebrochenes Ideal, so ist \(\ideal r\) endlich erzeugt, da
			\(A\) noethersch ist. Für jedes Primideal \(\ideal p \neq (0)\) ist
			\(\ideal r_{\ideal p} \neq (0)\) ein gebrochenes, also invertierbares
			Ideal im diskreten Bewertungsring \(A_{\ideal p}\). Damit ist
			\(\ideal r\) invertierbar.
		\item
			Sei umgekehrt jedes nicht verschwindende gebrochene Ideal invertierbar.
			Wie im letzten Beweis folgt, daß \(A\) noethersch ist. Damit reicht es
			zu zeigen, daß \(A_{\ideal p}\) für alle Primideal \(\ideal p \neq (0)\)
			ein diskreter Bewertungsring ist. Dazu reicht es zu zeigen, daß
			jedes (ganze) Ideal \(\ideal b \neq (0)\) in \(A_{\ideal p}\)
			invertierbar ist.
		\item
			Es ist aber \(\ideal a = A \cap \ideal b\) invertierbar, und damit
			auch \(\ideal b = \ideal a_{\ideal p}\).
			\qedhere
		\end{enumerate}
	\end{proof}
\end{frame}

\subsection{Anmerkungen zur Idealklassengruppe}

\begin{frame}{Idealgruppe eines Dedekindschen Bereiches}
	\begin{corollary}<+->
		In einem Dedekindschen Bereich \(A\) bilden die nicht verschwindenden
		gebrochenen Ideale eine Gruppe bezüglich der Multiplikation, die
		\emph{Idealgruppe von \(A\)}.
	\end{corollary}
	\begin{remark}<+->
		In diesem Zusammenhang können wir den Satz über die eindeutige
		Faktorisierbarkeit von nicht verschwindenden Idealen in Dedekindschen
		Bereichen in Primideale auch so formulieren:
		\\
		Die Idealgruppe eines Dedekindschen Bereiches ist eine freie abelsche
		Gruppe, welche durch die nicht verschwindenden Primideale von \(A\)
		erzeugt wird.
	\end{remark}
\end{frame}

\begin{frame}{Idealklassengruppe eines Dedekindschen Bereiches}
	Sei \(A\) ein Dedekindscher Bereich mit Quotientenkörper \(K\), dessen
	Idealgruppe \(\IdealG(A)\) sei.
	\\
	Wir haben einen Gruppenhomomorphismus \(K^\units \to \IdealG(A),
	u \mapsto (u)\), dessen Bild die gebrochenen Hauptideale sind.
	\begin{definition}<+->
		Die Faktorgruppe \(\ClassG(A) \coloneqq \IdealG(A)/K^\units\) heißt die
		\emph{Idealklassengruppe von \(A\)}.
	\end{definition}
	\begin{remark}<+->
		Im Kern von \(K^\units \to \IdealG(A)\) liegen genau die \(u \in K^\units\) mit
		\((u) = (1)\), also die Einheiten \(u \in A^\units\). Wir erhalten damit
		eine exakte Sequenz
		\[1 \to A^\units \to K^\units \to \IdealG(A) \to \ClassG(A) \to 1\]
		(multiplikativ geschriebener) abelscher Gruppen.
	\end{remark}
\end{frame}

\begin{frame}{Idealklassengruppe von Ringen ganzer Zahlen}
	Seien \(K\) ein Zahlkörper und \(A\) sein Ring ganzer Zahlen, ein
	Dedekindscher Bereich.
	\begin{remark}<+->
		In der algebraischen Zahlentheorie wird gezeigt, daß \(\ClassG(A)\)
		eine endliche Gruppe ist, ihre Ordnung heißt die \emph{Klassenzahl von
		\(K\)}.
		\\
		Es ist genau dann \([\ClassG(A) : 1] = 1\), wenn jedes invertierbare
		Ideal in \(A\) ein gebrochenes Hauptideal ist. Und dies ist genau dann
		der Fall, wenn \(A\) ein faktorieller Ring ist.
	\end{remark}
	\begin{remark}<+->
		Es ist \(A^\units\) eine endlich erzeugte abelsche Gruppe. Die Elemente
		endlicher Ordnung sind gerade die Einheitswurzeln \(\rou(K)\) von \(K\).
		Der Rang der freien abelschen Gruppe \(A^\units/\rou(K)\) ist \(r_1
		+ r_2 - 1\), wobei \(r_1\) die Anzahl der reellen und
		\(2 r_2\) die Anzahl der echt komplexen Einbettungen von \(K\) in \(\set C\)
		ist.
	\end{remark}
\end{frame}

\begin{frame}{Beispiele zu Ringen ganzer Zahlen}
	\begin{example}<+->
		Der Zahlkörper \(\set Q(\sqrt{-1})\) besitzt nur zwei echt komplexe Einbettungen.
		Sein Ring \(\set Z[i]\) ganzer Zahlen besitzt damit nur endlich viele Einheiten,
		nämlich alle Einheitswurzeln: \(\pm 1, \pm i\).
	\end{example}
	\begin{example}<+->
		Der Zahlkörper \(\set Q(\sqrt{2})\) besitzt nur zwei reelle Einbettungen.
		Sein Ring \(\set Z[\sqrt{2}]\) ganzer Zahlen hat damit unendlich viele
		Einheiten, und zwar \(\pm (1 + \sqrt{2})^n\) mit \(n \in \set Z\).
	\end{example}
\end{frame}

