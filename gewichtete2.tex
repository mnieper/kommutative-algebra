\section{Gewichtete Ringe und Moduln II}

\subsection{Exaktheit der Vervollständigung}

\begin{frame}{Exaktheit der Vervollständigung}
	\begin{proposition}<+->
		Sei \(A\) ein noetherscher kommutativer Ring. Sei \(0 \to M' \to M \to
		M'' \to 0\) eine exakte Sequenz endlich erzeugter \(A\)-Moduln.
		Für jedes \(\ideal a\) von \(A\) ist dann die induzierte Sequenz
		\(0 \to \hat M'_{\ideal a} \to \hat M_{\ideal a} \to \hat M''_{\ideal a}
		\to 0\) wieder exakt.
	\end{proposition}
	\begin{proof}<+->
		\begin{enumerate}[<+->]
		\item<.->
			Die \(\ideal a\)-adische Topologie auf \(M'\) ist nach dem 
			letzten Satz die von
			der \(\ideal a\)-adischen Topologie auf \(M\) induzierte. 
		\item
			Da \(\phi(\ideal a^n M) = \ideal a^n \phi(M) = \ideal a^n M''\)
			ist außerdem die \(\ideal a\)-adische Topologie auf \(M''\) die
			von der \(\ideal a\)-adischen Topologie auf \(M\) induzierte.
		\item
			Damit ist die Vervollständigung der exakten Sequenz bezüglich dieser
			Topologien wieder exakt.
			\qedhere
		\end{enumerate}
	\end{proof}
\end{frame}

\begin{frame}{Vervollständigung von Moduln}
	\begin{remark}<+->
		Sei \(A\) ein kommutativer Ring. Sei \(\ideal a\) ein
		Ideal von \(A\). Durch den kanonischen Homomorphismus \(A \to \hat A_{\ideal a}\)
		können wir die Vervollständigung in kanonischer Weise als \(A\)-Modul
		auffassen.
		\\
		Zu jedem \(A\)-Modul \(M\) können wir damit die Skalarerweiterung
		\(M_{\hat A_{\ideal a}}\) definieren. Der kanonische Homomorphismus
		\(M \to \hat M_{\ideal a}^A\) von \(A\)-Moduln definiert damit einen
		kanonischen \(\hat A_{\ideal a}\)-Modulhomomorphismus
		\[
			M_{\hat A_{\ideal a}}
			= M \otimes_A \hat A_{\ideal a} \to \hat M_{\ideal a} \otimes_A
			\hat A_{\ideal a} \to \hat M_{\ideal a} \otimes_{\hat A_{\ideal a}}
			\hat A_{\ideal a} \to \hat M_{\ideal a}.
		\]
		Für allgemeines \(A\) und \(M\) ist dieser kanonische Homomorphismus
		im allgemeinen weder injektiv noch surjektiv.
	\end{remark}
\end{frame}

\begin{frame}{Vervollständigung von Moduln als Skalarerweiterung}
	\begin{proposition}<+->
		\label{prop:compl_as_scalar_ext}
		Sei \(\ideal a\) ein Ideal in einem kommutativen Ring \(A\). Ist
		\(M\) ein endlich erzeugter \(A\)-Modul, so ist die kanonische
		Abbildung \(M_{\hat A_{\ideal a}} \to \hat M_{\ideal a}\) surjektiv.
		\\
		Ist \(A\) außerdem noethersch, so ist \(M_{\hat A_{\ideal a}} \to
		\hat M_{\ideal a}\) ein Isomorphismus.
	\end{proposition}
	\begin{proof}  <+->
		\begin{enumerate}[<+->]
		\item<.->
			Es ist leicht zu sehen, daß die \(\ideal a\)-adische
			Vervollständigung mit direkten Summen kommutiert. Für einen \(A\)-Modul
			der Form \(F \cong A^n\) gilt daher \(F_{\hat A} = F \otimes_A \hat A \cong \hat F\).
		\item
			Da \(M\) endlich erzeugt ist, existiert eine exakte Sequenz der
			Form \(0 \to N \to F \to M \to 0\) mit \(F \cong A^n\).
			\renewcommand{\qedsymbol}{}
			\qedhere
		\end{enumerate}
	\end{proof}
\end{frame}

\begin{frame}{Fortsetzung des Beweises zur Vollständigkeit von Moduln}
	\begin{proof}[Fortsetzung des Beweises]<+->
		\begin{enumerate}[<+->]
		\item<.->
			Die erste Zeile des kommutativen Diagrams
			\[
				\begin{CD}
					& & N_{\hat A} @>>> F_{\hat A} @>>> M_{\hat A} @>>> 0 \\
					& & @V{\gamma}VV @V{\beta}VV @V{\alpha}VV \\
					0 @>>> \hat N @>>> \hat F @>{\psi}>> \hat M @>>> 0
				\end{CD}
			\]
			ist exakt aufgrund der Rechtsexaktheit des Tensorproduktes.
		\item
			Es ist \(\psi\) surjektiv, da \(M\) die von \(F\) induzierte
			Topologie trägt. Da \(\beta\) surjektiv ist, folgt daraus die
			Surjektivität von \(\alpha\).
		\item
			Ist zusätzlich \(A\) noethersch, so ist \(N\) endlich erzeugt.
			Wir haben schon gezeigt, daß \(\gamma\) damit surjektiv ist.
			Außerdem ist dann die untere Zeile nach der letzten Proposition exakt.
		\item
			Eine Diagrammjagd zeigt, daß \(\alpha\) dann auch injektiv sein muß.
			\qedhere
		\end{enumerate}
	\end{proof}
\end{frame}

\begin{frame}{Flachheit der Vervollständigung}
	Sei \(A\) ein noetherscher kommutativer Ring. 
	\begin{proposition}<+->
		Für jedes Ideal
		\(\ideal a\) von \(A\) ist \(\hat A_{\ideal a}\) eine
		flache \(A\)-Algebra.
	\end{proposition}
	\begin{proof}<+->
		\begin{enumerate}[<+->]
		\item<.->
			Für endlich erzeugte \(A\)-Moduln \(M\) ist der Funktor
			\(M \mapsto \hat A \otimes_A M \cong \hat M\) exakt, da die
			Vervollständigung exakt ist.
		\item
			Wir haben schon allgemein gezeigt, daß dies Flachheit, also
			die Exaktheit für auch nicht endlich erzeugte Moduln, impliziert.
			\qedhere
		\end{enumerate}
	\end{proof}
	\begin{remark}<+->
		Für nicht endlich erzeugte \(A\)-Moduln ist der Funktor \(M \mapsto
		\hat M\) nicht exakt. Der gute Funktor ist daher \(M \mapsto M_{\hat A}\).
	\end{remark}
\end{frame}

\subsection{Vervollständigungen von Ringen}

\begin{frame}{Die Vervollständigung eines Ideals}
	\begin{proposition}<+->
		Sei \(\ideal a\) ein Ideal in einem noetherschen kommutativen Ring \(A\).
		Dann gilt für alle \(n \in \set N_0\):
		\begin{enumerate}[<+->]
		\item<.->
			\(\hat{\ideal a}_{\ideal a} = \hat A_{\ideal a} \ideal a
			\cong {\ideal a}_{\hat A_{\ideal a}}\).
		\item
			\(\widehat{\ideal a^n}_{\ideal a} = {\hat{\ideal a}_{\ideal a}}^n\).
		\item
			\(\ideal a^n/\ideal a^{n + 1}
			\cong \hat{\ideal a}_{\ideal a}^n/\hat{\ideal a}_{\ideal a}^{n + 1}\).
		\end{enumerate}
	\end{proposition}
	\begin{proof}<+->
		\begin{enumerate}[<+->]
		\item<.->
			Da \(A\) noethersch ist, ist \(\ideal a\) endlich erzeugt, womit
			die Abbildung \(\ideal a_{\hat A} = \ideal a \otimes_A \hat A \to
			\hat {\ideal a}\) ein Isomorphismus (mit Bild \(\hat A \ideal a\)) ist.
		\item
			\(\widehat{\ideal a^n} = \hat A \ideal a^n = (\hat A \ideal a)^n
			= \hat{\ideal a}^n\).
		\item
			\(\ideal a^n/\ideal a^{n + 1} = \ker(A/\ideal a^{n + 1} \to
			A/\ideal a^n) \cong \ker (\hat A/\hat{\ideal a}^{n + 1} \to
			\hat A/\hat{\ideal a}^n) = \hat{\ideal a}^n/\hat{\ideal a}^{n + 1}\).
			\qedhere
		\end{enumerate}
	\end{proof}
\end{frame}

\begin{frame}{Die Vervollständigung eines Ideals liegt im Jacobsonschen Ideal}
	\begin{proposition}<+->
		Sei \(\ideal a\) ein Ideal in einem noetherschen kommutativen Ring
		\(A\). Dann liegt \(\hat{\ideal a}_{\ideal a}\) im Jacobsonschen
		Radikal von \(\hat A_{\ideal a}\).
	\end{proposition}
	\begin{proof}<+->
		\begin{enumerate}[<+->]
		\item<.->
			Da \(\widehat{\ideal a^n} = \hat{\ideal a}^n\), ist
			\(\hat A\) vollständig bezüglich der \(\hat{\ideal a}\)-adischen
			Topologie.
		\item
			Für jedes \(x \in \hat{\ideal a}\) konvergiert
			\((1 - x)^{-1} = 1 + x + x^2 + \dotsb\) damit in \(\hat A\), so
			daß \(1 - x\) eine Einheit in \(\hat A\) ist.
		\item
			Damit liegt \(\hat{\ideal a}\) im Jacobsonschen Radikal von \(\hat A\).
			\qedhere
		\end{enumerate}
	\end{proof}
\end{frame}

\begin{frame}{Die Vervollständigung eines lokalen Ringes}
	\begin{proposition}<+->
		Sei \((A, \ideal m, F)\) ein noetherscher lokaler Ring. Dann ist
		\((\hat A_{\ideal m}, \hat{\ideal m}_{\ideal m}, F)\) ein lokaler Ring.
	\end{proposition}
	\begin{proof}<+->
		\begin{enumerate}[<+->]
		\item<.->
			Es ist \(\hat A/\hat{\ideal m} = A/\ideal m = F\), also ein Körper.  Damit ist
			\(\hat {\ideal m}\) ein maximales Ideal.
		\item
			Da \(\hat {\ideal m}\) im Jacobsonschen Ideal liegt und maximal ist, ist es das einzige
			maximale Ideal. Damit ist \(\hat A\) ein lokaler Ring.
			\qedhere
		\end{enumerate}
	\end{proof}
\end{frame}

\subsection{Der Krullsche Satz}

\begin{frame}{Krullscher Satz}
	\begin{theorem}[Krullscher Satz]<+->
		\label{thm:krull}
		Sei \(\ideal a\) ein Ideal in einem noetherschen kommutativen Ring \(A\). Sei
		\(M\) ein endlich erzeugter \(A\)-Modul. Der Kern \(K \coloneqq \bigcap\limits_{n = 1}^\infty
		\ideal a^n M\) des kanonischen Homomorphismus \(M \to \hat M_{\ideal a}\) besteht genau
		aus den \(x \in M\) mit \((1 + \ideal a) \cap \ann(x) \neq \emptyset\).
	\end{theorem}
	\begin{proof}<+->
		\begin{enumerate}[<+->]
		\item<.->
			Da \(K\) der Schnitt aller Umgebungen von \(0\) ist, ist die induzierte Topologie auf \(K\)
			die Klumpentopologie. Da die induzierte Topologie auch die \(\ideal a\)-adische ist,
			ist damit auch die \(\ideal a\)-adische Topologie auf \(K\)
			die Klumpentopologie, es ist also \(\ideal a K = K\).
		\item
			Damit existiert ein \(y \in \ideal a\) mit \((1 - y) K = 0\). 
		\item
			Ist umgekehrt \((1 - y) x = 0\) für ein \(y \in \ideal a\) und ein \(x \in M\), 
			so folgt \(x = y x = y^2 x = \dotsb \in K\).
			\qedhere
		\end{enumerate}
	\end{proof}
\end{frame}

\begin{frame}{Lokalisierung und Vervollständigung}
	\begin{remark}<+->
		Sei \(\ideal a\) ein Ideal in einem noetherschen kommutativen Ring \(A\). Sei \(S \coloneqq 1 + \ideal a\).
		Dann ist \(S\) multiplikativ abgeschlossen.
		\\
		Nach dem Krullschen Satz stimmen die Kerne der beiden kanonischen
		Abbildungen \(A \to \hat A_{\ideal a}\) und \(A \to S^{-1} A\) überein.
		\\
		Für jedes \(x \in \hat {\ideal a}_{\ideal a}\) gilt weiter \((1 - x)^{-1} = 1 + x + x^2 + \dotsb\), so daß
		jedes Element in \(S\) unter \(A \to \hat A_{\ideal a}\) zu einer Einheit wird.
		\\
		Die universelle Eigenschaft von \(S^{-1} A\) impliziert damit, daß \(S^{-1} A \to \hat A_{\ideal a},
		\frac a s \mapsto s^{-1} a\) zu einem wohldefinierten (injektiven) Homomorphismus von Ringen wird.
		\\
		Wir können (und werden) also \(S^{-1} A\) mit einem Unterring von \(\hat A_{\ideal a}\) identifizieren.
	\end{remark}
\end{frame}

\begin{frame}{Gegenbeispiel zum Krullschen Satz}
	\begin{example}<+->
		Ohne die Voraussetzung, daß der Ring noethersch ist, ist der Krullsche Satz im allgemeinen falsch.
		Sei etwa \(A\) der Ring aller \(\Cont^\infty\)-Funktionen auf \(\set R\) und \(\ideal a\) das Ideal der
		an \(0\) verschwindenden Funktionen (wegen \(A/\ideal a \cong \set R\) ist \(\ideal a\) maximal).
		\\
		Aus der Existenz der Taylorreihenentwicklung folgt, daß \(\ideal a\) von der identischen Funktion \(x\) erzeugt wird
		und daß \(\bigcap\limits_{n = 0}^\infty \ideal a^n\) die Menge aller \(f \in A\) ist, deren Taylorreihe in \(0\) trivial
		ist.
		\\
		Auf der anderen Seite ist \((1 + g) f = 0\) für ein \(g \in \ideal a\) genau dann Null, wenn \(f \in A\) einer ganzen
		Umgebung um \(0\) verschwindet.
		\\
		Damit liegt die Funktion \(\exp(-x^{-2})\) im Kern von \(A \to \hat A_{\ideal a}\) aber nicht im Kern von
		\(A \to (1 + \ideal a)^{-1} A\).
		\\
		Folglich ist \(A\) nicht noethersch.
	\end{example}
\end{frame}

\begin{frame}{Krullscher Schnittsatz}
	\begin{corollary}<+->
		\label{cor:krull}
		Sei \(A\) ein noetherscher Integritätsbereich. Für jedes echte Ideal \(\ideal a \neq (1)\) von \(A\) gilt
		dann \(\bigcap\limits_{n = 0}^\infty \ideal a^n = (0)\).
	\end{corollary}
	\begin{proof}<+->
		Nach dem Krullschen Satz ist \(\bigcap\limits_{n = 0}^\infty \ideal a^n\) der Kern von \(A \to (1 + \ideal a)^{-1} A\).
		Dieser ist trivial, da \(A\) ein Integritätsbereich ist.
	\end{proof}
\end{frame}

\begin{frame}{Hausdorffeigenschaft der \(\ideal a\)-adischen Topologie}
	\begin{corollary}<+->
		Sei \(A\) ein noetherscher kommutativer Ring. Sei \(\ideal a\) ein Ideal von \(A\), welches im Jacobsonschen
		Radikal von \(A\) enthalten ist. Für jeden endlich erzeugten \(A\)-Modul ist die \(\ideal a\)-adische Topologie auf
		\(M\) dann hausdorffsch.
	\end{corollary}
	\begin{visibleenv}<.->
		Es ist also \(\bigcap\limits_{n = 0}^\infty \ideal a^n M = (0)\).
	\end{visibleenv}
	\begin{proof}<+->
		Jedes Element der Form \(1 + x\) ist eine Einheit, wenn \(x\) in Jacobsonschen Radikal von \(A\) liegt.
	\end{proof}
	\begin{corollary}<+->
		Sei \((A, \ideal m)\) ein noetherscher lokaler Ring. Für jeden endlich erzeugten \(A\)-Modul ist die
		\(\ideal m\)-adische Topologie hausdorffsch. Insbesondere ist die \(\ideal m\)-adische Topologie auf \(A\)
		hausdorffsch.
		\qed
	\end{corollary}
\end{frame}

\begin{frame}{Der Schnitt aller primären Ideale zu einem Primideal}
	\begin{corollary}<+->
		Sei \(\ideal p\) ein Primideal in einem noetherschen kommutativen Ring \(A\). Dann ist der Kern des
		kanonischen Homomorphismus \(A \to A_{\ideal p}\) durch den Schnitt aller \(\ideal p\)-primären Ideale von \(A\)
		gegeben.
	\end{corollary}
	\begin{proof}<+->
		\begin{enumerate}[<+->]
		\item<.->
			Ist \(A\) lokal mit maximalem Ideal \(\ideal m\), so sind alle \(\ideal m\)-primären Ideale alle
			zwischen \(\ideal m\)
			und Idealen der Form \(\ideal m^n\) liegende Ideale. Damit zeigt die letzte Folgerung, daß der Schnitt
			aller \(\ideal m\)-primären Ideale in einem lokalen Ring das Nullideal ist.
		\item
			Ist \(A\) ein beliebiger kommutativer Ring erhalten wir, daß der Schnitt aller
			\(\ideal m \coloneqq A_{\ideal p} \ideal p\)-primären Ideale
			des lokalen Ringes \(A_{\ideal p}\) das Nullideal ist.
		\item
			Aus der Tatsache, daß die \(\ideal p\)-primären Ideale von \(A\) gerade die Kontraktionen der
			\(\ideal m\)-primären Ideale von \(A_{\ideal p}\) sind, erhalten wir die Behauptung.
			\qedhere
		\end{enumerate}
	\end{proof}
\end{frame}

